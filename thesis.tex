% per commentare una riga mettere % al suo inizio
% per s-commentare una riga (ossia attivarla) togliere il % al suo inizio
%
\documentclass[cucitura%lascia margine per la rilegatura
%,twoside% per stampa fronte-retro (fortemente consigliato per tesi voluminose, opzionale per le altre)
,12pt% font più grande (12pt) rispetto a quello normalmente usato (11pt)
]{toptesi}
%
% Cambiare encoding a piacere; oppure non caricare nessun encoding se si usano
% solo caratteri a 7 bit (ASCII) nei file d'entrata.
%
\usepackage[a-1b]{pdfx}% formato PDF/A, obbligatorio per l'archiviazione delle tesi di Polito
\usepackage[utf8]{inputenc}% IMPORTANTE! usare codifica UTF-8 per le lettere accentate
\usepackage{amsmath, amssymb}
\usepackage{nccmath}
\usepackage{appendix}
\usepackage{longtable}
\usepackage{lscape}
\usepackage{adjustbox}
\usepackage{float}
\usepackage{accsupp}
%
% Commentare le righe seguenti se NON si è specificata l'opzione "pdfa"
\hypersetup{%
    pdfpagemode={UseOutlines},
    bookmarksopen,
    pdfstartview={FitH},
    colorlinks,
    linkcolor={blue},
    citecolor={red},
    urlcolor={blue}
  }
% \documentclass[11pt,twoside,oldstyle,autoretitolo,classica,greek]{toptesi}
% \usepackage[or]{teubner}
%%%%%%%%%%%%%%%%%%%%%%%%%%%%%%%%%%%%%%%%%%%%%%%%%%%%
%


\input{commands.tex}

\definecolor{blond}{rgb}{0.98, 0.94, 0.75}
\definecolor{gray}{rgb}{0.4,0.4,0.4}
\definecolor{darkblue}{rgb}{0.0,0.0,0.6}
\definecolor{cyan}{rgb}{0.0,0.6,0.6}
\definecolor{Maroon}{rgb}{0.5,0.0,0.0}
\definecolor{darkgreen}{rgb}{0.0,0.5,0.0}

%\ExtendCaptions{english}{Abstract}{Acknowledgements}

\lstset{
	numbers=none, 
	numberstyle=\small, 
	numbersep=8pt, 
	frame = single, 
	framexleftmargin=20pt
}

\lstdefinelanguage{XML}
{
	backgroundcolor = \color{blond},
	basicstyle=\ttfamily\footnotesize,
	morestring=[b]",
	moredelim=[s][\bfseries\color{Maroon}]{<}{\ },
	moredelim=[s][\bfseries\color{Maroon}]{</}{>},
	moredelim=[l][\bfseries\color{Maroon}]{/>},
	moredelim=[l][\bfseries\color{Maroon}]{>},
	morecomment=[s]{<?}{?>},
	morecomment=[s]{<!--}{-->},
	commentstyle=\color{DarkOliveGreen},
	stringstyle=\color{blue},
	identifierstyle=\color{red}
}



\begin{document}
%\renewcommand{\lapagina}{\Roman{page}}


	\retrofrontespizio{Questo lavoro è soggetto a Licenza Creative Commons}
	\DottoratoIn{PhD Course in\space}
	\CorsoDiLaureaIn{Laurea Magistrale in}
	\NomeMonografia{Bachelor Degree Final Work}
	\TesiDiLaurea{Tesi di Laurea Magistrale}
	\NomeDissertazione{PhD Dissertation}
	\InName{in}
	\CandidateName{Candidate}
	\AdvisorName{Supervisors}
	\TutorName{Tutor}
	\NomeTutoreAziendale{Internship Tutor}
	\CycleName{cycle}
	\NomePrimoTomo{First volume}
	\NomeSecondoTomo{Second Volume}
	\NomeTerzoTomo{Third Volume}
	\NomeQuartoTomo{Fourth Volume}
	\logosede[8cm]{PolitoLogo3}% or comma separated list of logos


\ateneo{}

%%% scegliere la propria facoltà (solo PRIMA dell'AA 2012-2013)
%
%\facolta[III]{Ingegneria dell'Informazione}
%\facolta[IV]{Organizzazione d'Impresa\\e Ingegneria Gestionale}
%\Materia{Remote sensing}% uso sconsigliato

%\monografia{Gestione informatizzata di un magazzino ricambi}% per la laurea triennale
\titolo{Deployment automatico di funzioni di sicurezza di rete con Docker Compose}% per la laurea quinquennale e il dottorato
%\sottotitolo{Metodo dei satelliti medicei}% NON obbligatorio, per la laurea quinquennale e il dottorato

\corsodilaurea{Ingegneria Informatica (Computer Engineering)}% per la laurea di primo e secondo livello

\candidato{Benito \textsc{Marra}}% per tutti i percorsi
\relatore{prof.\ Fulvio Valenza}% per la laurea e il dottorato
\secondorelatore{prof.\  Riccardo Sisto}% per la laurea magistrale
\terzorelatore{\tabular[t]{@{}l}
	dott.  Daniele Bringhenti
	\endtabular}% per la laurea magistrale
%\sedutadilaurea{Agosto 1615}% per la laurea quinquennale
%\sedutadilaurea{\textsc{July} 2019}% per la laurea triennale
\sedutadilaurea{\textsc{Anno~Accademico} 2023-2024}% per la laurea magistrale
%\annoaccademico{1615-1616}% solo con l'opzione classica
%\annoaccademico{2006-2007}% idem

%\logosede{logopolito}
%
%\chapterbib %solo per vedere che cosa succede; e' preferibile comporre una sola bibliografia
%\AdvisorName{Supervisors}
%\newtheorem{osservazione}{Osservazione}% Standard LaTeX


\hypersetup{
   pdfpagemode={UseOutlines},
   bookmarksopen,
    pdfstartview={FitH},
    colorlinks,
    linkcolor={blue},
    citecolor={green},
    urlcolor={blue}
  }

%
% per numerare e far comparire nell'indice anche le sezioni di quarto livello
%\setcounter{secnumdepth}{4}% section-numbering-depth
%\setcounter{tocdepth}{4}% TOC-numbering-depth (TOC=Table-Of-nt)

%\setbindingcorrection{3mm}

\errorcontextlines=9

\expandafter\ifx\csname StileTrieste\endcsname\relax
    \frontespizio
\else
    \paginavuota
    \tomo
\fi




\sommario


Text of the summary 



\ringraziamenti

Ho sempre pensato che raggiungere un traguardo, seppur molto ambito e difficile, non ha valore se non sei circondato da persone con cui festeggiarlo. Ed è proprio
in casi come questo che mi sento fortunato, perchè fortunatamente ho tante persone intorno a me che mi supportano e sopratutto sopportano in tutte le scelte ed i percorsi che scelgo 
di percorrere. Probabilmente non riuscirò, date le mie pessime capacità come scrittore, a far emergere i sentimenti che provo nei vostri confronti ma comunque ci proverò.



A \textbf{Carolina}, grazie perchè è tutto merito tuo se la mia adolescenza e questi primi anni da giovane adulto sono stati riempiti da viaggi, ricordi, eventi e momenti indimenticabili. Grazie per tutte le volte
che hai organizzato qualcosa, che la tua voglia di far gruppo e di far festa mi ha contagiato ed è riuscita a portarmi leggerezza anche nei momenti più tesi della vita. Grazie anche per aver condiviso il peso 
della scrittura e del lavoro di tesi, e sopratutto grazie per aver quasi attentato alla mia vita facendomi stappare un prosecco in Svezia.

A \textbf{Federica}, che non saprei se definire più amica o moglie acquisita oramai. Fin da piccolo ho sempre sentito un ottimo feeling caratteriale nell'aprirmi e confidarmi con te. Grazie per non aver mai avuto una parola
fuori posto ed essere sempre in grado di comprendermi e aiutarmi anche nei momenti più bui della mia vita. Grazie perchè so che se mi sento perso e ti chiamo per chiedere aiuto anche nel cuore della notte saprai trovare la parola
corretta per aiutarmi. Grazie perchè la tua dolcezza non mi fa avere paura di mostrarmi fragile e vulnerabile, di poter mostrare l'affetto che provo nei tuoi confronti senza sentirmi a disagio.

A \textbf{Peppe Sergi}, grazie per la leggerezza e la complicità di questi anni. Grazie per le volte in cui mi hai fatto da spalla comica o mi hai sostenuto nelle discussioni nonostante i miei modi non propriamente appropriati,
ma sopratutto grazie per le numerose finanze che mi hai fatto risparmiare mettendo un freno alle idee di caro, mi hai salvato da una vita da mendicante.

A \textbf{Brubri}, grazie per essere stato sempre un punto di riferimento e motivo di fiducia mio e di tutto il resto del gruppo. Averti avuto finora nella mia vita mi ha insegnato contemporaneamente ad essere un bravo guidatore,
un bravo studente e un ottimo cuoco nelle rustute contemporaneamente. Spero di averti potuto ricambiare almeno un po’ mostrandoti come deve essere un buon marito, ti voglio bene.

A \textbf{Peppe Marra}, grazie per avermi dimostrato che anche le origini sono da commemorare. Senza di te a sollevare la mia visione della Calabria probabilmente sarei stato completamente distaccato dalla nostra terra. Grazie per
avermi mostrato tutto ciò che di bello abbiamo da offrire, dalle cene al satizzu ai posti di mare e montagna. Anche se ho avuto modo di legare con te temporalmente più tardi rispetto agli altri ragazzi del gruppo ti assicuro che l'affetto
che provo nei tuoi confronti è riuscito ad uguagliare subito quello di tutti gli altri.

A \textbf{Nicola Bova}, grazie per tutti i momenti di scontro passati assieme. Grazie perchè da questi piccoli modi diversi di vedere le varie cose della vita ne sono sempre uscito con un sorriso e arricchito nella diversità dei tuoi pensieri rispetto ai miei.
Grazie perchè seppur avendo pochissimo modo di vederci per via del lavoro e della distanza fisica, la tua presenza si sente sempre anche a distanza.

A \textbf{Delia}, dove onestamente i motivi per dire grazie sono talmente tanti che faccio veramente fatica ad elencarli tutti. Pensando al motivo più importante per cui sicuramente non passerai mai in secondo piano ai miei occhi è 
sicuramente per essere stata in grado di essere i punti che hanno richiuso le mie ferite. Grazie per aver condiviso la tua luce con me, sopratutto nel momento in cui non riuscivo a vedere altro che il colore nero intorno. 
Grazie anche per avermi spronato e insegnato a mettere me al primo posto, ad essere un po’ più egoista pur di prendermi cura delle mie esigenze, grazie per essere stata il motivo principale se il Benito adulto che sto diventando è così fiero di come è cresciuto. 
Spero che la nostra lontananza non ti abbia fatto pensare che la stima, l’ammirazione ed il bene che provo nei tuoi confronti sia diminuito, perché sarebbe un grande errore anche solo pensarlo, in quanto nel mio cuore ci sarà sempre un piccolo posto per la mia Deli.

A \textbf{Rosa}, trovare le parole per ringraziarti è stato probabilmente il compito più difficile di tutti. Essendo stata tu il prolungamento dei miei occhi, delle mie braccia e della mia bocca probabilmente sai già quanto tu sia importante nella mia vita, sia nella quotidianità che nei momenti più duri. 
Grazie perché mi hai dimostrato che l’amicizia tra uomo e donna esiste, che qualcuno possa diventare intimo con te  scoprendo fin nel profondo tutti gli scheletri nell’armadio (e sono veramente tanti) tanto da essere considerata una seconda sorella. 
Grazie perché mi hai aiutato a crescere, mi hai accompagnato a partire dall’adolescenza fino a tutti i drammi del diventare adulto. Sei riuscita ad essere contemporaneamente un educatrice che usa parole forti quando ne hai bisogno ed un affettuosa pseudo fidanzata quando ho avuto bisogno di affetto e supporto.
Grazie anche per tutte le infinite chiamate, le sbronze notturne condivise al telefono e le infinite passeggiate per tornare a casa dove si poteva parlare dalle cose più stupide a quelle più pericolose e potenzialmente incriminabili penalmente. Ma sopratutto grazie per avermi dimostrato che si può lottare con 
forza nonostante le malattie fisiche possano impedirci di godere appieno della vita, grazie di avermi fatto capire che essere un adulto responsabile significa anche non accontentarsi ma puntare sempre a ciò che pensiamo sia più giusto per noi. 

A \textbf{Nonno Benito}, so bene che non potrai leggere le parole che vorrei dirti ma un pensiero non potevo non dedicartelo. Più che un grazie vorrei dirti solo che ce l’ho fatta, che tuo nipote si sta costruendo il suo futuro passo dopo passo cercando di seguire le orme giganti che tu mi hai lasciato. 
Io non so se sarò in grado di raggiungere i livelli di bontà e amore che hai condiviso con gli altri ma sappi che ci sto provando con tutto me stesso obiettivo dopo obiettivo. Vorrei davvero potessi vedere tutte queste belle persone che mi hanno e continueranno a circondarmi, vorrei mostrarti il bene
che ricevo e che cerco di dare giornalmente, che sia con un sorriso o con delle parole di conforto. So per certo che la stella più luminosa in cielo sei sempre tu che dall’alto guardi come un secondo padre tutto ciò che mi capita, mi manchi veramente tanto.


%% inserire sempre nella tesi per la laurea di I livello, perché il nome dei tutori non è indicato sul frontespizio.
%Il lavoro descritto in questa monografia è stato svolto sotto la supervisione
%del Prof. Antonio Lioy (tutore accademico)% inserire sempre il nome del tutore accademico
% e dell'Ing. Mario Rossi (tutore aziendale)% inserire solo se la monografia è relativa ad un tirocinio.
%.

%\tablespagetrue % normalmente questa riga non serve ed e' commentata
%\figurespagetrue % normalmente questa riga non serve ed e' commentata

\indici

\listoffigures

\listoftables

\addcontentsline{toc}{chapter}{Listings}
\lstlistoflistings

\clearpage\pagestyle{empty}\mbox{}\clearpage

%\renewcommand{\lapagina}{\arabic{page}}

\mainmatter




\hypersetup{
    colorlinks=true,
    linkcolor=blue
}

\chapter{Introduzione} \label{ch:intro}

\section{Obiettivo della Tesi} 

Nell'ultimo decennio diverse tecnologie di rete si sono sviluppate,
creando reti sempre più robuste ed efficienti. In questo scenario,
una tecnologia in particolare, la \textit{"Network Function Virtualization"}(NFV)
ha reso possibile creare delle reti che svolgono le funzioni di sicurezza e
trasporto dei dati tramite la virtualizzazione di quest'ultime. Ciò ha permesso di definire
le \textit{"Software Defined Network"}(SDN) che introducono la possibilità di controllare le operazioni di rete
 tramite software.\\
Sfruttando queste tecnologie è stato sviluppato Verefoo(VErified REfinement and Optimized Orchestration) cioè 
un framework in grado di riuscire ad implementare nei nodi della rete i vari \textit{"Network Security Requirements"}(NSR) in una topologia predefinita e fornita
come input al framework.\\

\section{Descrizione della tesi}

Dopo aver spiegato nel Capitolo \hyperref[ch:intro]{[1]} gli obiettivi ed il lavoro prodotto per raggiungerli, il resto della
tesi è definito nel seguente modo:

\begin{itemize}
    \item Nella prima parte del Capitolo \hyperref[ch:verefoo]{[2]} si introduce il problema della Network Security Automation e si descrive il framework di Verefoo, ponendo particolare attenzione sul suo funzionamento ad alto e basso livello.
        Nella seconda parte sono descritte  le  definizioni delle Proprietà di sicurezza da passare come input al framework, con una spiegazione dettagliata
        di come queste intervengono nella definizione della topologia finale che verrà fornita come output dal framewrok. \\
        Infine verranno introdotti i grafi che verefoo richiede ed utilizza nella computazione dei vari \textit{NSF}.
    \item Il Capitolo \hyperref[ch:docker]{[3]} definisce l'architettura di docker, specificando la differenza tra usare docker per la virtualizzazione e delle semplici macchine virtuali. Successivamente
          viene fatto un approfondimento sul docker-compose, un tool in grado di poter istanziare più container velocemente tramite script. Nella parte finale viene spiegato come effettuare il networking
          sui container instanziati, come definirlo tramite docker-compose e come testare le comunicazioni in modo efficiente.
    \item Il Capitolo \hyperref[ch:ThesisObj]{[4]} descrive gli obiettivi posti all'inizio di questo lavoro di tesi. Più specificatamente, per ogni obiettivo presente verranno specificate le modalità e le scelte effettuate per portarlo a termine con una descrizione accurata dei vari passi che sono stati svolti prima della soluzione definitiva.
          Inoltre viene descritto in maniera più profonda rispetto a questo indice la descrizione dei futuri capitoli.
    \item Il Capitolo \hyperref[ch:ThesisObj]{[5]} descrive i lavori svolti nella prima delle due demo di cui questa tesi tratterà. Inizialmente viene descritto tramite pezzi di codice lo sviluppo dell'installer prodotto affinchè un qualsiasi utente possa
          utilizzare la demo in maniera pratica ed agile. Nei paragrafi successivi vengono evidenziati i punti critici incontrati, elencando le modifiche apportate affinchè essa possa funzionare correttamente.
          Nell'ultimo paragrafo infine verranno specificati ulteriori upgrade che si possono inserire nella demo per mettere in mostra in maniera ancora più evidente il lavoro svolto da Verefoo.
    \item Il Capitolo \hyperref[ch:intro]{[6]} descrive i lavori svolti ed implementati su Verefoo. In questo capitolo viene descritto il processo di merge fra le versioni precedentemente esistenti di Verefoo.
          Successivamente verrà quindi spiegato, anche tramite frammenti di codice, gli step
          che il framework eseguirà per produrre in output una rete che soddisfi contemporaneamente tutti i requisiti di sicurezza passati come input.  
          Infine si evidenziano anche le difficoltà che sono emerse lavorando al framework, e verranno proposte alcune soluzioni per poter evitare simili problematiche in futuro.
    \item Il Capitolo \hyperref[ch:intro]{[7]} descrive lo sviluppo della seconda Demo. In un primo momento viene mostrata la topologia
        di rete scelta da virtualizzare, con la finalità di indicare le nuove funzionalità di verefoo sviluppate al completamento del secondo obiettivo della tesi. Successivamente vengono descritti tutti i passi svolti per implementare la demo, con un commento per il codice che è stato utilizzato. Infine
        si evidenziano anche i limiti della demo prodotta con alcuni  futuri aggiornamenti possibili.
    \item Il Capitolo \hyperref[ch:conclusions]{[8]} elenca i lavori futuri da svolgere all'interno del framework, la necessità di poter implementare soluzioni
          alternative a quella proposta in questo documento, e i limiti che devono essere superati affinchè il framework possa essere utile in un ambiente reale e non
          solo di testing virtualizzato. Infine vengono descritte le conclusioni del lavoro, con un riassunto generale di tutto ciò che è stato prodotto.
\end{itemize}
%other chapters
\chapter{Background} \label{ch:verefoo}

\section{Introduzione a Verefoo} 

Descrivo il framework, cosa fa e quale è il suo scopo.

\section{Definizione dei Security Requirements}

Spiego in maniera breve come sono definiti i requisiti di sicurezza su verefoo

\section{Ambiente di test}

Spiego perchè è necessario avere un ambiente dove mostrare il framework in funzione


description \cite{noms2020}
\chapter{Docker} \label{ch:docker}

Negli ultimi anni numerose aziende che forniscono servizi agli utenti si sono trovate in serie difficoltà a causa dell' aumento sempre costante del numero
dei propri utenti e delle richieste di questi ultimi. Tali aziende si sono trovate quindi nella posizione di dover incrementare proporzionalmente le loro risorse disponibili, sia hardware che software, e di dover
assumere sempre più frequentemente del personale altamente specializzato per gestirle. Questa situazione di crisi ha portato ad un cambio di prospettiva in ambito aziendale, orientando l'attenzione verso un sistema di virtualizzazione
basato sull'utilizzo dei container invece che con le classiche macchine virtuali. Così facendo si è trovata una valida soluzione a questo problema, in quanto grazie alle loro caratteristiche i container consentono di garantire
la qualità del servizio offerto dalle aziende, ottimizzando l'utilizzo delle risorse hardware già in uso, senza la necessità di effettuare ulteriori investimenti significativi in hardware o personale aggiuntivo.\\
L'obiettivo di questo capitolo è di fornire una descrizione approfondita di una delle tecnologie più diffuse ed utilizzate in questo ambito, Docker\cite{docker-docs}. Inoltre viene anche descritto l'utilizzo di uno strumento, docker-compose, molto utile per 
creare e gestire applicazioni multi-container. Nella parte finale del capitolo è infine possibile trovare una descrizione e degli esempi su come eseguire il networking all'interno dei container.

\section{Differenze fra Container e Macchine Virtuali} 

Prima di scendere nello specifico descrivendo il funzionamento di docker all'interno di un sistema operativo è necessario definire nel dettaglio cosa sia una macchina virtuale e cosa un container e perchè è più efficiente la seconda soluzione rispetto alla prima.\\
Per quanto riguarda le macchine virtuali una definizione ci viene fornita dal sito ufficiale di VMware \cite{vmware}:\\
"Una Macchina Virtuale (VM) è una risorsa di elaborazione che utilizza software al posto di un computer fisico per eseguire programmi e distribuire applicazioni. Una o più macchine virtuali (guest) vengono eseguite su una macchina fisica (host). 
Ogni macchina virtuale esegue il proprio sistema operativo e funziona in modo separato dalle altre VM, anche quando sono tutte in esecuzione sulla stessa macchina host. Ciò significa che, ad esempio, una macchina virtuale con sistema operativo MacOS può essere eseguita su un PC fisico."\\

Sotto la prospettiva dei container, è possibile ottenere una definizione precisa consultando la documentazione ufficiale di Docker\cite{docker-container}.\\
"Un container è  un'unità standard di software che raggruppa il codice e tutte le sue dipendenze in modo che l'applicazione possa essere eseguita rapidamente e in modo affidabile da un ambiente di calcolo all'altro."\\

Seppur le due definizioni possano sembrare molto simili, la struttura software che sta dietro ad entrambe le tecniche di virtualizzazione è diversa, come viene mostrato nella seguente figura:\\


\begin{figure}[h]  % 'h' significa che la figura viene posizionata qui
    \centering
    \includegraphics[width=1\textwidth]{VMvsContainers.png}  % Sostituisci 'nome_immagine' con il nome del tuo file immagine e l'estensione
    \caption{Architettura Virtual Machine e Containers}
    \label{fig:VMvsContainers}
\end{figure}
 
Le macchine virtuali hanno infatti una struttura più complessa rispetto ai container. È infatti presente un hypervisor\cite{hypervisor}, che è un software che permette di creare e gestire le macchine virtuali in maniera veloce, efficiente, flessibile e portabile.
Sopra all'hypervisor ogni macchina virtuale ha il suo proprio sistema operativo, costringendo l'host OS ad allocare numerose risorse per istanziare anche solo 1 macchina virtuale. Si può notare come questa soluzione sia difficlmente scalabile perchè
ciò porta il server hardware sul quale poggia tutto il sistema ad essere ulteriormente stressato con l'aggiunta di ogni macchina virtuale. Viceversa, i container richiedono solo le risorse minime necessaria a fare funzionare l'applicazione di turno, installando solo i file binari e le librerie necessarie.\\
Più in generale le caratteristiche vantaggiose dei container rispetto alle macchine virtuali sono riassunte dalle seguenti parole chiave:\\


\begin{itemize}
    \item \textbf{Modularità}: avere la possibilità di creare un container per ogni possibile task permette di suddividere i container in vari moduli, ognuno che svolge una sepcifica funzione del progetto di riferimento. Operando in tale direzione, è possibile sviluppare progetti con un approccio Bottom-Up
        , portando ad un ambiente di testing e validation più veloce ed immediato sui singoli moduli.
    \item \textbf{Isolamento}: ogni container che esegue una immagine viene visto come un ambiente isolato, indipendente dagli altri container in esecuzione. Questo approccio semplifica notevolmente l'individuazione di possibili bug ed errori nel progetto.
    \item \textbf{Peso in Memoria}: come analizzato nel lavoro di Martin Lindström\cite{performance-container}, differentemente dalle macchine virtuali, i container sono delle virtualizzazioni molto più leggere e che richiedono meno risorse alla macchina ospitante. Questo è derivato dal fatto che i container contengono solo lo stretto necessario all'applicazione per funzionare correttamente,
        mentre le VM hanno bisogno anche di istanziare un proprio sistema operativo, che richiede una discreta quantità di spazio.
    \item \textbf{Scalabilità}: Avendo un peso molto ridotto rispetto alle macchine virtuali, la necessità di aumentare le performance e le dimensioni di un progetto trova nei container un ottimo fattore di scalabilità. È  infatti possibile scalare i sistemi sia verticalmente, perchè all'aumentare delle risorse del sistema operativo ospitante corrisponde un aumento della velocità di reazione dei container
    che orizzontalmente, perchè aggiungere una feature corrisponde nell'aggiungere un container al sistema già funzionante. 
    \item \textbf{Condivisone Risorse}: Attraverso i file di configurazione dei container è possibile condividere file con il container, che nell'ambiente isolato verranno considerate come risorse dedicate, anche se in realtà sono condivise. Ciò estende questa funzionalità se si mettono in comune le stesse risorse per più container. In questo caso sulle risorse verrà messo un lock che bloccherà le risorse fino a che
        uno dei container non abbia finito di utilizzarle, rilasciandole.
    \item \textbf{Fast Boot}:Non dovendo dipendere da nessun sistema operativo, i container possono avviarsi ed essere operativi molto più velocemente rispetto alle macchine virtuali.
    \item \textbf{Operazioni su disco}: Avendo un collegamento diretto con il sistema operativo le operazioni su  disco (scrittura, lettura e cancellazione) sono più veloci, portando ad un aumento delle performance per processi parallelizzabili.
\end{itemize}

\newpage

\section{Docker e la sua architettura}

Tra le piattaforme software disponibili per istanziare e gestire containers per applicazioni Docker riveste il ruolo di software leader nel settore.
Nato nel 2013 e progettato da Solomon Hykes nell'azienda dotCloud, Docker è un progetto open source. La differenza fondamentale dagli altri software è che
basa la maggior parte delle operazioni eseguibili su un demone chiamato appunto Docker, che svolge le operazioni di istanziazione dei container e la loro gestione.
Al fine di garantire ciò, il demone richiede come file di input delle \textit{"Immagini"}. Come è descritto nella documentazione di Docker\cite{docker-container}: 
"Un'immagine di container Docker è un pacchetto leggero, autonomo ed eseguibile di software che include tutto il necessario per eseguire un'applicazione: codice, runtime, strumenti di sistema, librerie di sistema e impostazioni."\\

L'architettura di Docker sfrutta il meccanismo client-server, della quale viene mostrata una rappresentazione:\\

\begin{figure}[h]  % 'h' significa che la figura viene posizionata qui
    \centering
    \includegraphics[width=1\textwidth]{Docker_Architecture.png}  % Sostituisci 'nome_immagine' con il nome del tuo file immagine e l'estensione
    \caption{Architettura Docker}
    \label{fig:DockerArchitecture}
\end{figure}

Di seguito una spiegazione di tutti gli elementi che intervengono nella creazione di un container:

\begin{itemize}
    \item \textbf{Docker Client}: Rappresenta la macchina fisica nel quale è installato Docker. Può comunicare con il controller principale (Docker Host) tramite delle Rest API.
    Le chiamate che il client può effettuare sono le seguenti:
        \begin{itemize}
            \item \textit{Docker Pull}: viene utilizzato per scaricare un'immagine da un Registro. Il demone controllerà se questa immagine è presente nel registro locale indicato, altrimenti andrà a cercare online
                la versione più recente dal registro predefinito di Docker Hub.
            \item \textit{Docker Build}: questo comando permette la creazione di un'immagine dato un file di configurazione definito dall utente (deve avere il nome di Dockerfile) e una cartella di riferimento.
            \item \textit{Docker Run}: questa chiamata fa creare al demone un container con l'immagine specificata da linea di comando ed avvia il container.
        \end{itemize}
    \item \textbf{Docker Deamon}: è il cuore dell'architettura di Docker. Il ruolo del docker deamon(anche chiamato dockerd) è di ascoltare le richieste tramite call API del client e gestire il registro Docker dove sono contenute
        le immagini, i plugin e le estensioni. Inoltre può comunicare con altri demoni per gestire i servizi Docker.
    \item \textbf{Docker Registry}: è  una zona di memoria che memorizza le immagini Docker. In questa zona sono anche presenti eventuali plugin installati su Docker e le estensioni sviluppate per sfruttare i servizi di Docker.
        Talvolta potrebbe capitare che le immagini ricercate nel registro non sono presenti, ed in questo caso viene effettuato un collegamento diretto con un registro pubblico online definito Docker Hub per poter usufruire di alcune
        immagini già pronte.
\end{itemize}


Ha senso inserire anche come si scrive un dockerfile con esempi di comando o no?

\section{Il tool Compose}
Come si è potuto notare nella sezione precedente, Docker garantisce un grado di flessibilità molto elevato che consente di creare sia container che svolgono ruoli molto semplici che container più articolati che richiedono anche l'installazione di diverse
librerie tramite Dockerfile. Tuttavia può capitare molto spesso che l'ambiente isolato di un container all'altro sia una limitazione in quanto per svolgere determinati task due o più macchine virtuali devono potere comunicare efficacemente, inoltre la gestione
di reti complesse tramite singoli Dockerfile e linea di comando può risultare tediosa e molto scomoda da utilizzare. Basti pensare che nel caso di un container non funzionante bisognerebbe modificare non solo il Dockerfile del singolo elemento, ma anche rieseguire tutte le chiamate
di sistema per fare rebuild dell'ambiente virtuale. \\
Per venire incontro a questi problemi molto comuni nello sviluppo di applicazioni aziendali, Docker propone delle soluzioni innovative e che cercano non solo di risolvere i problemi sopracitati , ma anche di introdurre dei meccanismi di semplificazione per la gestione di reti complesse.




\section{Networking}
\chapter{Obiettivi della tesi} \label{ch:ThesisObj}

Questo capitolo introduce gli obiettivi di questa tesi, descrivendo studi e metodologie utilizzate al fine di raggiungerli.\\
Nei capitoli precedenti è stato infatti descritto lo stato dell'arte di Docker e Verefoo, che sono i due strumenti principali utilizzati per
svolgere questo lavoro di tesi. Il primo è infatti uno strumento fondamentale per poter garantire un ambiente di testing efficiente e isolato, il secondo
invece è il framework principale nel quale la quasi totalità del lavoro si è svolta. Comprendere l'utilizzo correto dei due elementi è quindi di fondamentale importanza al fine di 
poter capire, continuare e migliorare lo stato attuale di Verefoo. Inoltre, molti dei risultati prodotti precedentemente su Verefoo rispetto a questo lavoro pur essendo corretti non
offrivano alcun modo di mostrare in maniera diretta le innovazioni prodotte ai nuovi utenti che si approcciavano al framework. Parte del lavoro svolto è quindi basato sull'ideare e produrre
dei metodi efficaci e semplici per mostrare all'utente le capacità e caratteristiche di Verefoo.\\
Entrando più nello specifico, gli obiettivi della tesi possono essere definiti dal seguente elenco:


\begin{enumerate}
    \item Come primo obiettivo ci si è focalizzati su una demo già presente all'interno dell'ecosistema. All'interno di questa, tuttavia, diversi elementi all'interno erano considerabili obsoleti
        o scorretti, di conseguenza ci si è posti come scopo principale di questa prima parte correggere e perfezionare la demo per mostrare correttamente le potenzialità del framework.
        Allo stato iniziale, il framework era in grado di accettare solo un determinato requisito di sicurezza di rete ovvero
        la \textit{Protection Property} cioè la possibilità di far passare il traffico crittografato da un nodo ad un altro della topologia
        in maniera sicura. Al fine di garantire ciò vengono allocati nella topologia dei VPN Gateway in grado di poter cifrare il traffico in ingresso e decifrare quello in uscita.
        La topologia proposta utilizzerà uno scenario verosimile a quello che ci si potrebbe aspettare in un'azienda di piccole-medie dimensioni, nella quale al fine di poter garantire
        la correttezza dei requisiti proposti, verranno istanziati 6 VPN Gateway. I lavori svolti per questo obiettivo sono consultabili nel capitolo 5 di questa tesi.\newpage
    \item Una volta terminato il restauro della demo sulle VPN, è emersa la necessità di integrare alle funzionalità già presenti la possibilità di configurare anche i packet filter. Come secondo obiettivo ci si è quindi concentrati per trovare una soluzione al fine di poter integrare le varie versioni di Verefoo. 
        Inizialmente il framework era diviso in differenti branch, due dei quali permettevano rispettivamente l'allocazione solamente dei VPN Gateway o dei Firewall configurati come packet filter per garantire la \textit{Isolation Property} e la \textit{Reachability Property}.
        Il traguardo previsto è quello di creare un ulteriore branch che permettesse la fusione dei due precedentemente descritti. Per ottenere ciò diverse soluzioni sono state esplorate. Inizialmente si è pensato di avere una soluzione mista tramite due versioni del framework attive contemporaneamente che comunicavano fra loro in sequenza,
        per poi passare a soluzioni che permettevano con un solo file jar di svolgere entrambe le funzioni in una sola esecuzione. Anche in questo caso sono stati analizzate entrambe le possibili soluzioni per implementare questo obiettivo, sia istanziando prima i Firewall che i gateway VPN che il viceversa.
        La soluzione finale scelta è stata quella di allocare prima i VPN Gateway e successivamente i Firewall, con delle motivazioni a supporto che verranno estese nel capitolo 6.
    \item Concluso il lavoro sul framework è risultato essenziale trovare un modo per mostrare i risultati ottenuti. L'ultimo obiettivo del lavoro svolto è stato quindi la progettazione, lo sviluppo e l'implementazione di un'altra demo, diversa dalla precedente che mostrasse le nuove potenzialità del framework. \\
        A differenza della prima, che da questo momento verrà definita come Demo-A, la seconda, che chiameremo Demo-B, propone un esempio di topologia di rete molto più complessa e con diverse proprietà di sicurezza aggiuntive. Lo sviluppo di questa ha richiesto, come nella precedente, la realizzazione di un'ambiente virtuale dedicato creato con
        Docker-Compose nel quale mostrare come le varie proprità venissero rispettate. Infine, per agevolare i futuri lavori nel framework è stato prodotto in linguaggio Bash un installer per rendere semplice ed immediato l'installazione del framework. Ulteriori approfondimenti sul codice e le scelte effettuate sono descritte nel capitolo 7.
\end{enumerate}

Come ultima appendice al lavoro svolto ai fini di questa tesi, è infine presente una breve conclusione del lavoro che oltre a fare un riassunto generale sugli obiettivi raggiunti definisce i futuri lavori possibili e suggerisce anche alcuni aggiornamenti e perfezionamenti che possono essere svolti nelle demo e nel framework prodotti.


\lstdefinestyle{bashstyle}{
    language=bash,
    basicstyle=\small\ttfamily,
    backgroundcolor=\color{gray!10},
    keywordstyle=\color{blue},
    commentstyle=\color{green!50!black},
    stringstyle=\color{red},
    showstringspaces=false,
    morekeywords={mkdir, ls, cd, mv, rm, chmod, sudo}
}

\lstdefinestyle{yaml}{
     basicstyle=\color{red}\footnotesize,
     rulecolor=\color{black},
     string=[s]{'}{'},
     stringstyle=\color{red},
     comment=[l]{:},
     commentstyle=\color{black},
     morecomment=[l]{-}
 }

\chapter{Correzione, sviluppo e ottimizzazione Demo A} \label{ch:DemoA}

All'interno di questo capitolo sarà fornita una descrizione completa dei lavori svolti nel primo progetto di demo che è stato preso in analisi inizialmente, corretto e migliorato successivamente.
\\ Inizialmente viene descritta con una breve introduzione gli obiettivi che ci si aspettava di raggiungere con lo sviluppo della demo, i problemi presenti all'inizio del lavoro svolto e le soluzioni possibili attuabili.
\\ Nella seconda parte viene trattato, entrando più nello specifico, l'implementazioni delle soluzioni proposte e presentato il lavoro finale. 
\\ Nell'ultima parte del capitolo viene infine mostrato una prova di correttezza della demo con la verifica dei suoi output e delle sue funzionalità.


\section{Introduzione alla Demo}
Come descritto al Capitolo[\ref{ch:verefoo}] Verefoo è un framework in grado di definire dei requisiti di sicurezza ad alto livello, allocare in maniera ottimale varie Network Security Functions (NSF) all'interno della topologia di rete fornita in input e configurare automaticamente tali funzioni 
automaticamente. Allo stato attuale il framework è ancora in fase di sviluppo e, nonostante si ponga gli obiettivi appena descritti, non tutte le funzionalità sono attualmente possibili all'interno del framework. Più precisamente, varie versioni del framework sono presenti in stato di sviluppo, e ognuna si occupa di allocare una possibile NSF separatamente alle altre.
All'interno di questo capitolo verrà presa in considerazione per il lavoro una di queste possibili versioni, ovvero quella che si occupa della verifica dei Network Security Requirements di protezione, dell'allocazione del numero minimo ottimale di VPN Gateway per rispettare i requisiti in input, e della configurazione di questi ultimi.\\
L'obiettivo principale di questo lavoro di demo è quindi mostrare all'utente come questa versione sia in grado, data una topologia di rete simile a quelle di una azienda di medie dimensioni, di verificare i requisiti, allocare le VPN e configurarle correttamente.
\newpage
Un altro elemento emerso durante i lavori sullo sviluppo di questa demo è stata la difficile accessibilità di quest'ultima. Per far funzionare sia il framework che la demo sono infatti necessari numerosi tool da installare all'interno della macchina in alcune versioni specifiche e potrebbe essere non immediato installare correttamente il dispositivo per far funzionare 
framework e demo. Di conseguenza come obiettivo secondario, ma di uguale importanza è stato prodotto un installer che permette all'utente di ottenere automaticamente tutti i programmi nelle versioni corrette per poter utilizzare il framework a proprio piacimento.
\\
Per portare a termine questi due obiettivi ci si è quindi interfacciati con un progetto di demo che risultava incompleto e mal funzionante. Analizzando più approfonditamente possiamo dividere le modifiche effettuate all'interno di questa demo nei seguenti punti:

\begin{enumerate}
    \item \textbf{Versione Framework}: Il file del framework iniziale era malfunzionante, in quanto ogni qual volta che si provava ad avviarlo il terminale segnalava il file come corrotto ed inutilizzabile.
    \item \textbf{Chiamate API}: La demo utilizzava delle chiamate API considerabili obsolete, di conseguenza qualsiasi relazione con il framework non produceva alcun risultato.
    \item \textbf{Installer}: Come accennato precedentemente la mancanza di un installer della Demo rendeva il framework poco accessibile e di difficile installazione manuale.
    \item \textbf{Certificati VPN}: Alcuni dei certificati di chiavi pubbliche e private erano ormai scaduti, sono quindi stati sostituiti ed i file di configurazione modificati opportunatamente.
    \item \textbf{Docker Compose}: Il file di configurazione dell'ambiente virtuale di Docker Compose conteneva errori e molti dei container non venivano istanziati correttamente.
    \item \textbf{Forwarding Rules}: Alcune forwarding rules all'interno dell'ambiente virtuale erano scorrette, sono state quindi corrette ed aggiornate per garantire la comunicazione fra tutti i nodi della rete.
\end{enumerate}

    
\section{Sviluppo Installer}
\label{sec:Installer}
Come menzionato nell'elenco precedente, uno dei punti fondamentali dei lavori effettuati su questo progetto è stata la programmazione e implementazione di un installer all'interno della demo.\\
Verefoo è un framework che per funzionare necessita le seguenti specifiche:
\begin{itemize}
    \item \textbf{Linux Ubuntu 20.04 Long Term Support(LTS)} come sistema operativo della macchina ospitante il framework.
    \item \textbf{Pv} come programma per monitorare i dati mandati attraverso la pipe.
    \item \textbf{Docker Engine} come tecnologia per istanziare e gestire container virtuali.
    \item \textbf{Docker-Compose v1} come plugin di Docker per poter istanziare e gestire molteplici container utilizzando un unico file di configurazione.
    \item  \textbf{Java openjdk-1.8} per poter eseguire il framework correttamente.
    \item \textbf{Curl} come programma per poter effettuare chiamate API con il framework.
    \item \textbf{Z3 4.8.15} come programma per risolvere il problema MaxSMT che Verefoo crea durante la computazione dell'output.
\end{itemize} 

Tuttavia al fine di poter implementare tutti questi elementi tramite terminale linux non è sufficiente installare i vari packages tramite il comando
\textit{"apt-get install [packageName]"} ma è anche necessario modificare opportunatamente alcune variabili d'ambiente all'interno del sistema operativo. Inoltre molti di questi package devono essere installati con delle versioni specifiche per evitare problemi di compatibilità,
di conseguenza per l'utente può risultare ostico riuscire a reperire ed installare tutte le versioni correttamente in quanto molti software non sono aggiornati alle versioni più recenti ma bisogna ricercare la versione specifica sui github dei vari programmi, rendendo quindi il comando apt-get install inutilizzabile.
\\ \\
Di conseguenza è stato prodotto uno script bash che da questo momento in poi verrà denominato installer che si occupa di controllare se i package sono già effettivamente presenti all'interno del sistema operativo. In caso affermativo si controlla se la versione del package è quella corretta per utilizzare Verefoo, altrimenti verrà
effettuato un downgrade della versione corrente. In caso negativo il package verrà invece istanziato direttamente.
\\Di seguito vengono mostrati e commentati alcuni snippet di codice appartententi all'installer:

\begin{lstlisting}[style=bashstyle, caption={Installazione packages curl e pv}, label=lst:bash-example,numbers=left]
    #!/bin/bash
    if which pv &> /dev/null
    then
    printf "${GREEN}pv installed... OK${COLOR_RESET}\n"
    else
    printf "${BLUE}pv package not installed. I'm going to install it\n${COLOR_RESET}\n"
    sudo apt-get update
    sudo apt-get --yes install pv
    printf "${GREEN}pv installed... OK${COLOR_RESET}\n"
    fi
    if which curl &> /dev/null
    then
    printf "${GREEN}curl installed... OK${COLOR_RESET}\n"
    else
    printf "${BLUE}curl package not installed. I'm going to install it\n${COLOR_RESET}\n"
    sudo apt-get update
    sudo apt-get --yes install curl
    printf "${GREEN}curl installed... OK${COLOR_RESET}\n"
    fi
\end{lstlisting}

Per quanto riguarda i package pv e curl l'installazione è piuttosto semplice, infatti si può notare come sia alla riga 2 che alla riga 11 venga controllato che i comandi pv e curl siano presenti reinderizzando l'output sia in caso di successo che di errore al folder /dev/null che è un folder nel quale i 
dati vengono cancellati automaticamente ma che restuituisce il successo o il fallimento dell'operazione. Nel caso l'operazione restituisca successo vuol dire che il package è già presente all'interno del sistema operativo, in caso contrario viene installato eseguendo precedentemente una \textit{apt-get update}
per scaricare localmente la versione più aggiornata del package.
Per quanto riguarda l'installazione di java invece l'installazione è leggermente più complessa:

\begin{lstlisting}[style=bashstyle, caption={Installazione java openjdk}, label=lst:bash-example,numbers=left]
    VERSION=$(java -version 2>&1 >/dev/null | grep "java version\|openjdk version")
    if [ "" = "$VERSION" ]; then
    printf "${BLUE}Java-openjdk not installed. I'm going to install it\n${COLOR_RESET}\n"
    sudo apt-get update
    sudo apt-get --yes install openjdk-8-jdk 
    echo "JAVA_HOME=/usr/lib/jvm/java-1.8.0-openjdk-amd64/jre" | sudo tee -a /etc/environment
    printf "${GREEN}Java installed... OK${COLOR_RESET}\n"
    else
    printf "${GREEN}Java openjdk installed... OK${COLOR_RESET}\n"
    fi
\end{lstlisting}

In questo caso per controllare se la versione corretta di java sia già installata viene invocato il comando \textit{"java -version"} effettuando un redirect dell'output
e intercettando delle righe contententi le stringhe "java version" o "openjdk version" assegnando questo risultato alla variabile VERSION. \\
Successivamente viene controllato se il risultato dato da questo comando corrisponde alla stringa vuota o se effettivamente è stata trovata una versione di java.
Nel caso in cui java non sia presente all'interno del sistema operativo viene installato tramite il comando \textit{"apt-get install openjdk-8-jdk"} e infine viene definito il path dove il sistema operativo dovrà andare
a cercare le varie librerie di sistema ogni qualvolta un comando java verrà eseguito, che viene salvato all'interno del file contentente tutte le variabili d'ambiente, che si trova al path:\textit{"/etc/environment"}. 

\begin{lstlisting}[style=bashstyle, caption={Installazione Docker e Docker Compose}, label=lst:bash-example,numbers=left]
    if [ -x "$(command -v docker)" ]; then
    printf "${GREEN}docker installed... OK${COLOR_RESET}\n"
    else
     printf "${BLUE}Docker engine not installed. I'm going to install it\n${COLOR_RESET}\n"
    sudo apt-get update
    sudo apt-get install ca-certificates curl gnupg
    sudo install -m 0755 -d /etc/apt/keyrings
    curl -fsSL https://download.docker.com/linux/ubuntu/gpg | sudo gpg --dearmor -o /etc/apt/keyrings/docker.gpg
    sudo chmod a+r /etc/apt/keyrings/docker.gpg
    echo \
  "deb [arch="$(dpkg --print-architecture)" signed-by=/etc/apt/keyrings/docker.gpg] https://download.docker.com/linux/ubuntu \
  "$(. /etc/os-release && echo "$VERSION_CODENAME")" stable" | \
  sudo tee /etc/apt/sources.list.d/docker.list > /dev/null
  sudo apt-get --yes install docker-ce docker-ce-cli containerd.io docker-buildx-plugin docker-compose-plugin
    printf "${GREEN}docker engine installed installed... OK${COLOR_RESET}\n"
    fi
    if [ -x "$(command -v docker-compose)" ]; then
    printf "${GREEN}docker-compose installed... OK${COLOR_RESET}\n"
    else
    printf "${BLUE}Docker-compose package not installed. I'm going to install it\n${COLOR_RESET}\n"
    sudo apt-get update
    sudo apt-get --yes install docker-compose
    printf "${GREEN}docker-compose installed... OK${COLOR_RESET}\n"
    fi

\end{lstlisting}

In questa parte di codice viene installato Docker ed il suo plugin docker-compose. Similmente all'installazione di curl e pv 
viene controllata l'esistenza di entrambi alle righe 1 e 17 e successivamente vengono installati utilizzando la documentazione ufficiale di 
Docker\cite{dockerubuntu} e di Docker-Compose\cite{dockercomposelinux}

\begin{lstlisting}[style=bashstyle, caption={Installazione Z3}, label=lst:bash-example,numbers=left]
    FILE=/home/z3
    if [ -d "$FILE" ]; then
    printf "${GREEN}z3 exists in /home directory... OK${COLOR_RESET}\n"
    else
    printf "${BLUE}z3 doesnt exist in home directory, im going to install it\n${COLOR_RESET}\n"
    cd /home
    sudo curl -LO  https://github.com/Z3Prover/z3/releases/download/
                    z3-4.8.15/z3-4.8.15-x64-glibc-2.31.zip
    sudo unzip z3-4.8.15-x64-glibc-2.31.zip
    sudo rm z3-4.8.15-x64-glibc-2.31.zip 
    sudo mv z3-4.8.15-x64-glibc-2.31 z3    #rename z3-4.8.15 into z3 
    echo "LD_LIBRARY_PATH=$LD_LIBRARY_PATH:/home/z3/bin/" | sudo tee -a /etc/environment # Care!! >> not > or it will over write environment variables
    sudo echo "Z3=/home/z3/bin/" | sudo tee -a /etc/environment
    printf "${GREEN}Z3 installed and environment variables setted... OK${COLOR_RESET}\n"
    fi
\end{lstlisting}
Come ultimo passo l'installer controlla l'esistenza del tool di Z3 all'interno del sistema operativo. Diversamente dai packages standard installati finora, per potersi far rilevare correttamente dal framework
è necessario che Z3 sia installato nella Home directory con un path definito come \textit{"/home/z3"}. Conseguentemente per controllare l'esistenza del tool è necessario quindi verificare solo la presenza del folder z3 nella home directory.
In caso di installazione necessaria tramite una chiamata API con curl viene scaricato dal github ufficiale di Z3 \cite{z3prover} la versione desiderata di Z3, estratta nella home directory e rinominata in \textit{"z3"}.
Infine vengono istanziate 2 variabili d'ambiente ovvero \textit{"LD-LIBRARY-PATH"} e \textit{"Z3"} all'interno del file con le variabili d'ambiente che si trova al path \textit{"/etc/environment"}.

\section{Implementazione}
Il secondo e più corposo lavoro all'interno di questa prima parte della tesi è stata l'effettiva realizzazione della Demo A tramite ambiente virtuale. L'obiettivo principale che ci si è posti da raggiungere è stato di mostrare le potenzialità di Verefoo proponendo all'utente una topologia di rete 
che si avvicinasse il più possibile a quella di una rete di un'azienda di piccole dimensioni. Per raggiungere tali scopi la soluzione che è stata implementata è la seguente:

\begin{figure}[h]  % 'h' significa che la figura viene posizionata qui
    \centering
    \includegraphics[width=1\textwidth]{VPN_AG.PNG} 
    \caption{Service Graph Demo A}
    \label{fig:ServiceGraph}
\end{figure}
Come è possibile notare, la topologia proposta presenta sulla sinistra 3 WebServer (e1,e2,e3) il cui traffico viene gestito da un load balancer (lb1) il quale si occupa di evitare congestioni di traffico durante le comunicazioni fra clients e servers. 
Congiuntamente ai WebServer sono presenti anche 5 endpoints (e4,e5,e6,e7,e8) che all'interno dell'ambiente virtuale verranno istanziati come fossero dei WebClients. 
Al fine di poter testare il corretto funzionamento del framework all'interno della rete sono anche presenti dei nodi che fungeranno da monitor, come il nodo  s11, ed altri che invece svolgeranno la semplice funzione di forwarder con le rispettive static routes, per simulare dei router generici. 
Infine sono presenti diversi nodi al momento vuoti che rappresentano gli  allocation places che Verefoo richiede fra i suoi possibili input per velocizzare il processo di risoluzione del problema MaxSMT. All'interno di questi nodi, il framework potrà collocare delle network security functions per assicurarsi 
il corretto funzionamento dei requisiti di sicurezza all'interno della rete.\\ Di seguito viene fornita una tabella con la definizione di ogni nodo, del suo indirizzo IP che verrà utilizzato nell'ambiente virtuale e della sua funzionalità all'interno della topologia. \\

\begin{table}
    \centering
    \begin{tabular}{ccc}
        \hline
         Name & IP & Functionality \\
        \hline
        e1 & 130.10.0.1 & Web servers behind load balancer b1 \\
        e2 & 130.10.0.2 & * \\
        e3 & 130.10.0.3 & * \\
        e4 & 40.40.41.1 & Web Client \\ 
        e5 & 40.40.42.1 & Web Client \\
        e6 & 88.80.84.1 & Web Client \\
        e7 & 192.168.1.1 & Web Client \\
        e8 & 192.168.2.1 & Web Client \\
        lb1 & 130.10.0.4 & Load Balancer \\
        s10 & 33.33.33.2 & Web Cache \\
        s11 & 33.33.33.3 & Forwarder \\
        s12 & 220.124.30.1 & Forwarder \\
        s13 & 33.33.33.4 & Forwarder \\
        a7 & 1.0.0.7 & Forwarder \\
        \hline
    \end{tabular}
    \caption{Node definitions and functionalities}
    \label{tab:tabella}
\end{table}


Oltre alla definizione del Service Graph per la demo è necessario fornire al framework anche l'insieme di security requirement che la rete deve avere, ovvero una traduzione di tutte le prioprietà di sicurezza che vorremmo fossero presenti all'interno della nostra azienda.\\
Per poter simulare il più possibile un'azienda reale è stato stabilito di creare numerosi requisiti di protezione con l'obiettivo di avere una topologia di output che contenesse un numero elevato di VPN Gateway (6), in quanto è molto comune anche in ambienti di smart working avere dei tunnel dedicati per ogni Host che lavora all'esterno della rete aziendale.. 
Al fine di poter ottenere una topologia simile sono state definite le seguenti regole:\\


\begin{table}[H]
    \centering
    \small
    \setlength{\tabcolsep}{1pt} % Riduci il padding delle colonne
    \begin{tabular}{ccccccccc}
        \hline
         Policy & IPSrc & IPDst & pSrc & pDst & tProto & Confidentiality & Intregrity & Untrusted nodes\\
        \hline
        Protection & 40.40.41.1 & 130.10.0.1 & * & 22 & ANY & AES-256-CBC & SHA2-256 & 33.33.33.2 \\
        Protection & 88.80.84.1 & 130.10.0.* & * & 80 & ANY & AES-256-CBC & SHA2-256 & 33.33.33.2/33.33.33.3 \\
        Protection & 192.168.1.1 & 130.10.0.1 & * & * & ANY & AES-256-CBC & SHA2-256 & 33.33.33.2/33.33.33.3 \\
        Protection & 40.40.42.1 & 192.168.2.1 & * & * & ANY & AES-256-CBC & SHA2-256 & 33.33.33.4/220.124.30.1 \\
        \hline
    \end{tabular}
    \caption{Security Requirements Definition DemoA}
    \label{tab:tabella}
\end{table}
\begin{itemize}
    \item \textbf{Prima Regola}: Il Web Client e4 deve poter comunicare in maniera sicura con il Web Server e1. Il traffico originato da e4 può utilizzare qualsiasi porta di uscita per il protocollo di trasporto ma il Server e1 deve ricevere i dati in ingresso unicamente dalla porta 22. È possibile utilizzare sia il protocollo UDP che TCP per il trasporto. Viene specificato infine il nodo s10 come nodo non sicuro e attraverso il quale il traffico deve passare cifrato.
    \item \textbf{Seconda regola}: Il Web Client e6 deve poter comunicare in maniera sicura con tutti i Web Server (e1,e2,e3). Il traffico originato da e6 può utilizzare qualsiasi porta di uscita per il protocollo di trasporto ma i Server devono ricevere i dati in ingresso solo dalla porta 80. È possibile utilizzare sia il protocollo UDP che TCP per il trasporto. In questo caso i nodi considerati non sicuri sono s10 e s11.
    \item \textbf{Terza Regola}: Il Web Client e7 deve poter comunicare in maniera sicura con il Web Server e1. Non ci sono limitazioni sulle porte per il traffico in entrata ed in uscita e può essere utilizzato qualsiasi protocollo di quarto livello per il trasporto. I nodi non sicuri, come per la seconda regola, sono s10 e s11. 
    \item \textbf{Quarta Regola}: Il Web Client e5 deve poter comunicare in maniera sicura con il Web Client e8. Come per la terza regola, non ci sono limitazioni sulle porte e protocollo di trasporto da utilizzare. I nodi considerati non sicuri per questa regola sono s12 ed s13.
\end{itemize}

Definire questi elementi all'interno del framework di Verefoo non è comunque sufficiente per creare una demo, in quanto come già visto al Capitolo[\ref{ch:verefoo}] è sufficiente descrivere in file XML la definizione dei nodi ed i corrispettivi indirizzi ip e nodi adiacenti. \\
Per avere una demo che invece dimostri la correttezza delle operazioni del framework è necessario istanziare gli elementi descritti in output e successivamente tradurli in un ambiente virtuale di container Docker. Di seguito è quindi disponibile la definizione di alcuni services
per la topologia descritta in precedenza:

\begin{lstlisting}[style=yaml,caption={Definizione di Services dell'ambiente virtuale DemoA},label=composeDemoA]
    services:
        server1:
            container_name: server1
            hostname: server1
            image: endpoint
            cap_add:
            - NET_ADMIN
            command: sh -c "route del default && route add -net 0.0.0.0 netmask 0.0.0.0 gw 130.10.0.4 && tail -F anything"
            networks:
                servers:
                    ipv4_address: 130.10.0.1
    
        end4:
            container_name: end4
            hostname: end4
            image: endpoint
            cap_add:
                - NET_ADMIN
            command: sh -c "route del default && route add -net 0.0.0.0 netmask 0.0.0.0 gw 40.40.41.100 && tail -F anything"
            networks:
                endpoints4:
                    ipv4_address: 40.40.41.1
    
\end{lstlisting}

Scendendo nel dettaglio sono mostrati principalmente un server ed un endpoint, che appartengono a sottoreti differenti ma che hanno la stessa immagine di partenza nella costruzione del container.
Entrambi infatti verranno configurati come endpoint e le loro forwarding route verranno cancellate completamente, indicando come unico gateway un ip stabilito arbitrariamente nella sottorete in cui appartengono.
Inoltre tramite il cap-add viene dato a ciascun endpoint la capacità di NET-ADMIN che consente al container di configurare l'interfaccia di rete e le route.

\section{Output}
Fornendo gli input definiti precedentemente il framework cercherà una soluzione che non solo soddisfi tutte le regole, ma che impieghi anche il minor numero di risorse necessarie per garantire tali proprietà. Verefoo risolverà quindi un problema di tipo MaxSMT (Maximum Satisfiability Modulo Theories). 
Nel caso specifico di questa topologia con i requisiti di sicurezza previamente definiti nel paragrafo precedente il risultato prodotto in output sarà il seguente:
\begin{figure}[h]  % 'h' significa che la figura viene posizionata qui
    \centering
    \includegraphics[width=1\textwidth]{VPN_deploy.PNG} 
    \caption{Verefoo Output Demo A}
    \label{fig:VPNDeployA}
\end{figure}

Come era prevedibile, diversi VPN Gateway sono stati allocati al posto dei vari Allocation Places definiti nell'input fornito a Verefoo. Data questa soluzione che dovrebbe essere la soluzione al problema definito prima, proveremo a testare dentro l'ambiente virtuale la funzionalità dei vari tunnel VPN, per assicurarci che il framework lavori correttamente.
Prima di procedere però è importante anche analizzare i vari elementi che Verefoo ha prodotto. Il framework infatti non ha solo allocato negli allocation places corretti i Gateway, ma ha anche fornito anche una configurazione automatica da utilizzare. Nel caso specifico la configurazione per i 6 Gateway è la seguente:\\

\begin{table}[H]
    \centering
    \begin{tabular}{ccccccc}
        \hline
         \# & Action & IPSrc & IPDst & pSrc & pDst & tProto \\
        \hline
        1 & EXIT & 192.168.1.1 & 130.10.0.1 & * & * & ANY \\
        2 & EXIT & 88.80.84.1 & 130.10.0.3 & * & 80 & ANY \\
        3 & EXIT & 40.40.41.1 & 130.10.0.1 & * & 22 & ANY \\
        4 & EXIT & 88.80.84.1 & 130.10.0.1 & * & 80 & ANY \\
        5 & EXIT & 88.80.84.1 & 130.10.0.2 & * & 80 & ANY \\
        6 & ACCESS & 130.10.0.1 & 192.168.1.1 & * & * & ANY \\
        7 & ACCESS & 130.10.0.3 & 88.80.84.1 & 80 & * & ANY \\
        8 & ACCESS & 130.10.0.1 & 40.40.41.1 & 22 & * & ANY \\
        9 & ACCESS & 130.10.0.1 & 88.80.84.1 & 80 & * & ANY \\
        10 & ACCESS & 130.10.0.2 & 88.80.84.1 & 80 & * & ANY\\
        \hline
    \end{tabular}
    \caption{VPN Gateway 1}
    \label{tab:VPN Gateway 1}
\end{table}

\begin{table}[H]
    \centering
    \begin{tabular}{ccccccc}
        \hline
         \# & Action & IPSrc & IPDst & pSrc & pDst & tProto \\
        \hline
        1 & ACCESS & 40.40.41.1 & 130.10.0.1 & * & 22 & ANY \\
        2 & EXIT & 130.10.0.1 & 40.40.41.1 & 22 & * & ANY \\
        \hline
    \end{tabular}
    \caption{VPN Gateway 2}
    \label{tab:VPN Gateway 2}
\end{table}

\begin{table}[H]
    \centering
    \begin{tabular}{ccccccc}
        \hline
         \# & Action & IPSrc & IPDst & pSrc & pDst & tProto \\
        \hline
        1 & ACCESS & 192.168.1.1 & 130.10.0.1 & * & * & ANY \\
        2 & EXIT & 130.10.0.1 & 192.168.1.1 & * & * & ANY \\
        \hline
    \end{tabular}
    \caption{VPN Gateway 3}
    \label{tab:VPN Gateway 3}
\end{table}

\begin{table}[H]
    \centering
    \begin{tabular}{ccccccc}
        \hline
         \# & Action & IPSrc & IPDst & pSrc & pDst & tProto \\
        \hline
        1 & ACCESS & 192.168.2.1 & 40.40.42.1 & * & * & ANY \\
        2 & EXIT & 40.40.42.1 & 192.168.2.1 & * & * & ANY \\
        \hline
    \end{tabular}
    \caption{VPN Gateway 4}
    \label{tab:VPN Gateway 4}
\end{table}

\begin{table}[H]
    \centering
    \begin{tabular}{ccccccc}
        \hline
         \# & Action & IPSrc & IPDst & pSrc & pDst & tProto \\
        \hline
        1 & ACCESS & 40.40.42.1 & 192.168.2.1 & * & * & ANY \\
        2 & EXIT & 192.168.2.1 & 40.40.42.1 & * & * & ANY \\
        \hline
    \end{tabular}
    \caption{VPN Gateway 5}
    \label{tab:VPN Gateway 5}
\end{table}

\begin{table}[H]
    \centering
    \begin{tabular}{ccccccc}
        \hline
         \# & Action & IPSrc & IPDst & pSrc & pDst & tProto \\
        \hline
        1 & ACCESS & 88.80.84.1 & 130.10.0.1 & * & 80 & ANY \\
        2 & ACCESS & 88.80.84.1 & 130.10.0.2 & * & 80 & ANY \\
        3 & ACCESS & 88.80.84.1 & 130.10.0.3 & * & 80 & ANY \\
        4 & EXIT & 130.10.0.1 & 88.80.84.1 & 80 & * & ANY \\
        5 & EXIT & 130.10.0.2 & 88.80.84.1 & 80 & * & ANY \\
        6 & EXIT & 130.10.0.3 & 88.80.84.1 & 80 & * & ANY \\
        \hline
    \end{tabular}
    \caption{VPN Gateway 6}
    \label{tab:VPN Gateway 6}
\end{table}

Prendendo in esempio il VPN Gateway 1 è possibile notare come Verefoo non si limita ad eseguire delle configurazioni semplici ma è in grado di produrre anche configurazioni miste, ovvero non unicamente di ingresso o di uscita dal tunnel. 
È infatti possibile notare come tutto il traffico in transito può sia cifrato in Accesso (ACCESS) al tunnel che decifrato in uscita (EXIT) al tunnel.\\
Oltre alle configurazioni dei gateway VPN, Verefoo offre, tramite un traduttore automatico, dei file di configurazione per istanziare i tunnel VPN utilizzando Strongswan.\\
A fine di esempio viene fornito il file di configurazione di Strongswan di uno dei vari gateway della topologia, che verrà utilizzato per istanziare il tunnel VPN nell'ambiente virtuale:\\


\begin{lstlisting}[language=sh]
    connections {
    site-site {
      local_addrs  = 20.0.1.1
      remote_addrs = 20.0.7.2
      local {
         auth = pubkey
         certs = VpnConfig1Cert.pem
         id = VpnConfig1.strongswan.org
      }
      remote {
         auth = pubkey
         id = VpnConfig2.strongswan.org
      }
      children {
         net-net {
            local_ts = 130.10.0.1/24            
            remote_ts = 192.168.1.1/24            
            start_action = trap|start 
            rekey_time = 5400
            rekey_bytes = 500000000
            rekey_packets = 1000000
            esp_proposals = aes256-sha2_256-modp2048
         }
      }
      version = 2
      mobike = no
      reauth_time = 10800
    }
    }
\end{lstlisting}

\section{Verifiche e Test}
Definito l'ambiente virtuale e impostato le configurazioni di Verefoo prodotte in output è necessario testare e dimostrare che il framework ha prodotto una soluzione corretta.\\
Al fine di eseguire tutte le operazioni di verifica e test in maniera più trasparente possibile è stato inserito, in aggiunta alla topologia già definite nella figura (\ref{fig:VPNDeployA})
un nodo aggiuntivo, di collegamento fra il vpn gateway 4 e l'endpoint 8. 
In questo modo, utilizzando tcpdump e analizzando le interfacce di rete sarà possibile notare in quali nodi della topologia il traffico passerà in chiaro e in quali invece verrà cifrato. 
È importante sottolineare come all'interno dell'ambiente virtuale tutti gli elementi sono stati configurati precedentemente per velocizzare le operazioni di testing, di conseguenza i vari certificati pubblici e privati sono stati già generati ed inseriti all'interno dei nodi VPN, e le route di trasmissione statiche sono state già inserite per tutta la topologia tramite il file di docker-compose.\\

All'interno di questa tesi, per semplificare le operazioni di verifica effettuate non verranno verificati tutti i requisiti di sicurezza definiti precedente ma ci si limiterà a testare la connessione fra l'endpoint e5 e l'endpoint e8.\\

Per iniziare il test di correttezza sul tunnel vpn che collega i due endpoint è necessario attivare strongswan con i certificati dei due gateway vpn, utilizzando quindi il comando \textit{"swanctl -q"} all'interno del container di vpn4 e vpn5:

\begin{figure}[h]
    \begin{minipage}{0.5\textwidth}
        \centering
        \includegraphics[width=\linewidth]{(01)FirstVPNConfig.png}
        \caption{VpnGateway4}
    \end{minipage}\hfill
    \begin{minipage}{0.5\textwidth}
        \centering
        \includegraphics[width=\linewidth]{(02)SecondVPNConfig.png}
        \caption{VpnGateway5}
    \end{minipage}
\end{figure}

\newpage

Una volta caricati correttamente i vari certificati all'interno dei container, è necessario utilizzare strongswan per creare il tunnel esplicitamente. Per fare ciò è necessario invocare il seguente comando:
\textit{swanctl --initiate --child net-net}. Attraverso l'output mostrato su terminale è possibile verificare che i tunnel siano stati correttamente istanziati o se vi è stato qualche problema nello scambio dei parametri
di sicurezza.
\begin{figure}[h] 
    \centering
    \includegraphics[width=1\textwidth]{(03)VPNConnectionEstablished.png} 
    \caption{Verifica Tunnel VPN}
    \label{fig:VPNDeploy}
\end{figure}

Come si può notare dall'output stampato sul terminale il protocollo IPsec utilizza IKE per scambiarsi le chiavi e successivamente viene stabilita la security association 
definita "net-net" nel file di configurazione swanctl. Una volta che viene stabilita la security association viene definito il traffic selector, ovvero si specifica quali pacchetti dovranno
entrare nel tunnel VPN (con rispettivo IPsrc ed IPdst). Infine vengono scelti gli algoritmi per l'autenticazione e la cifratura dei pacchetti. Se entrambi i nodi riescono ad accettare le stesse condizioni allora il tunnel viene creato.
Nel caso in esempio quindi come si può leggere nell'output il tunnel è stato configurato da Verefoo correttamente.
\\
Da questo momento in poi quindi qualsiasi pacchetto inviato dall'endpoint e8 all'endpoint e5 e viceversa verrà cifrato e non sarà possibile ispezionarlo all'interno del tunnel VPN. Per verificare ciò verranno utilizzati due monitor, 
il primo definito monitorESP sarà fornito dal container del nodo s12, mentre il secondo definito monitorICMP sarà il nodo precedentemente aggiunto fra l'endpoint e8 e il gateway vpn4. In questo modo è possibile utilizzare tcpdump e osservare le interfacce di rete nelle quali passano i pacchetti fra e8 ed e5.  
Per quanto riguarda il monitorESP la corretta interfaccia è eth1 mentre per il monitorICMP è eth0. 
Eseguiamo quindi il seguente comando: \textit{ tcpdump -i [interfacename]} su entrambi i monitor per osservare il traffico dati.

\begin{figure}[h] 
    \centering
    \includegraphics[width=1\textwidth]{(04)MonitorsConfig.png} 
    \caption{Tunnel stabilito}
    \label{fig:VPNDeploy}
\end{figure}

Una volta configurato tcpdump in entrambi i monitor tutti i pacchetti in transito attraverso questi due nodi verranno registrati e controllati, indicando ip e protocollo di livello 3 che viene utilizzato.
Per testare che i pacchetti vengano effettivamente cifrati correttamente quindi mandiamo dei pacchetti di ping da e8 ad e5 e controlliamo l'output dei due monitor:

\begin{figure}[H] 
    \centering
    \includegraphics[width=1\textwidth]{(05)Correctness Proof.png} 
    \caption{Prova di correttezza}
    \label{fig:VPNDeploy}
\end{figure}

Osservando l'ultimo output si può dedurre che i tunnel funzionano correttamente.
Per il monitorICMP che si trova fra l'endpoint e il vpn gateway 4 il traffico viene trasmesso in chiaro ed i pacchetti vengono visualizzati come pacchetti ICMP da tcpdump, invece i pacchetti che transitano su s12 cioè dopo il passaggio dal vpn gateway vengono visualizzati come pacchetti ESP che è il protocollo utilizzato da IPsec per incapsulare i dati nei tunnel. Di conseguenza l'output prodotto da verefoo soddisfa le regole fornite ed utilizza il minor numero di risorse allocate possibili, verificando la correttezza del framework.\\
Con questo output è possibile quindi affermare il corretto funzionamento del framework e della Demo A e, di conseguenza, il raggiungimento del primo obiettivo di questo lavoro di tesi.
\chapter{Merge} \label{ch:MergeChapter}

Merge works of the 2 verefoo versions
\lstdefinestyle{yaml}{
     basicstyle=\color{red}\footnotesize,
     rulecolor=\color{black},
     string=[s]{'}{'},
     stringstyle=\color{red},
     comment=[l]{:},
     commentstyle=\color{black},
     morecomment=[l]{-}
 }

 \lstdefinestyle{bashstyle}{
    language=bash,
    basicstyle=\small\ttfamily,
    backgroundcolor=\color{gray!10},
    keywordstyle=\color{blue},
    commentstyle=\color{green!50!black},
    stringstyle=\color{red},
    showstringspaces=false,
    morekeywords={mkdir, ls, cd, mv, rm, chmod, sudo}
}

\chapter{Design, progetto e sviluppo Demo B} \label{ch:DemoB}
All'interno di questo capitolo viene descritto il deisgn, lo sviluppo e la realizzazione del secondo lavoro di Demo prodotto.
Questo lavoro si differenzia dal precedente descritto al Capitolo [\ref{ch:DemoA}] in quanto ogni elemento appartenente alla Demo è stato creato da zero,
senza avere nessun riferimento precedente.\\
Inizialmente verrà fornita una breve introduzione alla demo, descrivendo obiettivi e motivazioni che han portato allo sviluppo, successivamente invece verrà
descritta la topologia di rete progettata e i requisiti di sicurezza richiesti, analizzando le motivazioni che hanno portato alla scelta di questi ultimi.\\
Terminata l'introduzione verrà poi descritto lo sviluppo dell'ambiente virtuale, facendo riferimenti a snippet di codici effettivamente implementati all'interno dell'ambiente e 
descrivendo accuratamente i file utilizzati per creare immagini e container all'interno di Docker Compose.\\
Infine l'ultima parte del capitolo, in maniera simile al capitolo sulla Demo A, proporrà dei test e delle verifiche della correttezza degli output forniti da Verefoo.

\section{Introduzione}
Come è stato ampiamente discusso nel Capitolo [\ref{ch:MergeChapter}] la nuova versione di Verefoo prodotta è ora in grado di allocare, in un'unica iterazione, contemporanemante due tipi di Network Security Functions:
i Firewall configurati come dei Packet Filter e dei VPN Gateway che consentono l'autenticazione e la cifratura dei pacchetti durante le comunicazioni fra due endpoint. Tramite queste novità è quindi possibile definire contemporanemante
sia le proprietà di isolamento e raggiungibilità che quelle di protezione, garantendo flessibilità all'utente. Essendo questo un risultato definibile come una milestone nel percorso di sviluppo di Verefoo si è pensato di realizzare una Demo che
possa mostrare i progressi raggiunti tramite l'istanziazione di un nuovo ambiente virtuale. In questo caso rispetto al precedente non ci si è concentrati sul definire dei requisiti il più simile possibile a quelli di una ipotetica azienda di medie dimensioni
quanto più a mostrare in maniera evidente come tutti e 3 i requisiti di sicurezza vengono rispettati. Nonostante questo obiettivo si è però deciso di utilizzare una topologia più complessa rispetto a quella della Demo A, così da mostrare anche come con diversi elementi
che aumentano il carico computazionale del framework, la soluzione che viene fornita in output è computata in maniera rapida, ottimale e sopratutto corretta. \\
Avendo già sviluppato un installer per la Demo A (paragrafo \ref{sec:Installer}) all'interno del repository non è stato necessario crearne uno nuovo, in quanto i package utilizzati in questa Demo sono gli stessi di quella precedente, è quindi stato fornitto all'utente lo stesso 
file.

\section{Implementazione}
Per raggiungere gli obiettivi descritti nell'introduzione si è pensato di creare molti più host rispetto alle versioni precedenti. In questo modo è possibile notare anche le potenzialità di virtualizzazione che l'utilizzo dei container tramite Docker Compose ci permette di avere.
Inoltre ogni host che viene rappresentato in figura non rappresenterà un unico elemento all'interno della topologia ma uno dei possibili host della sottorete definita nel quadrato in cui l'host è contenuto. Di seguito viene quindi fornita una rappresentazione grafica della topologia: 
\begin{figure}[h]  % 'h' significa che la figura viene posizionata qui
    \centering
    \includegraphics[width=1\textwidth]{Allocation_Graph.jpg} 
    \caption{Grafo di Allocazione della Demo B}
    \label{fig:AllocationGraphB}
\end{figure}

Entrando più nel dettaglio è possibile notare diversi gruppi di host, che rappresentano delle ipotetiche sedi aziendali separate. La prima, in alto a sinistra, viene descritta con la sottorete 20.1.0.0/16. All'interno di questo cluster sono presenti 
tre sottoreti rappresentate dagli ip 20.1.1.0/24, 20.1.2.0/24, 20.1.3.0/24. Le comunicazioni di queste sottoreti vengono regolate da un Load Balancer, che in caso di congestione del traffico decide arbitrariamente in quale nodo della rete inoltrare il traffico
per decongestionare alcuni collegamenti. La seconda sede aziendale è rappresentata in basso a sinistra dalla rete 20.2.0.0/16. Come per la prima sede sono presenti 4 sottoreti di host definite dagli ip 20.2.1.0/24, 20.2.2.0/24, 20.2.3.0/24, 20.2.4.0/24. Diversamente dalla 
prima, all'esterno del cluster 20.2.0.0/16 è presente un NAT che si occupa di mascherare gli indirizzi ip all'interno della rete con il resto della topologia. In basso sono inoltre presenti altre due sedi più piccole delle precedenti, definite dalle sottoreti 20.3.1.0/24 
e 20.4.1.0/24, queste dovrebbero simulare eventuali dipendenti in smart working che quindi non si connettono da una grande rete aziendale ma dalla propria rete di casa privata.\\
Sulla destra è invece presente una Server Farm definita dalla rete 10.0.0.0/24; all'interno sono stati definiti due Web Server con IP 10.0.0.1 e 10.0.0.2.  Infine è presente un nodo che funge da monitor (Mnt) per ispezionare il traffico in entrata ed uscita dai Web Server.
La topologia appena descritta può essere sintetizzata dalla tabella seguente:
\begin{table}
    \centering
    \begin{tabular}{ccc}
        \hline
         Name & IP & Functionality \\
        \hline
        C1-1 & 20.1.1.1 & Web Client behind load balancer \\
        C1-2 & 20.1.2.1 & Web Client behind load balancer \\
        C1-3 & 20.1.3.1 & Web Client behind load balancer \\
        Lb & 33.33.33.1 & Load balancer \\ 
        C2-1 & 20.2.1.1 & Web Client behind NAT \\
        C2-2 & 20.2.2.1 & Web Client behind NAT \\
        C2-3 & 20.2.3.1 & Web Client behind NAT \\
        C2-4 & 20.2.4.1 & Web Client behind NAT \\
        FW1 & 33.33.33.3 & Forwarder \\
        C3-1 & 20.3.1.1 & Web Client \\
        C4-1 & 20.4.1.1 & Web Client \\
        Mnt & 33.33.33.2 & Traffic Monitor \\
        S1 & 10.0.0.1 & Web Server \\
        S2 & 10.0.0.2 & Web Server \\
        \hline
    \end{tabular}
    \caption{Definizione nodi della topologia per Demo B}
    \label{tab:tabella}
\end{table}

Entrambi i server sono i punti d'interesse più importanti della topologia perchè le comunicazioni che verranno effettuate durante la simulazione utilizzeranno come 
uno dei due host almeno uno dei due server. Entrando più nello specifico l'obiettivo dei requisiti di sicurezza di isolamento e raggiungibilità sarà quello di rendere le 
comunicazioni della prima sede possibili solo con il server 10.0.0.2, della seconda sede solo con il server di 10.0.0.1 e delle restanti due sedi con entrambi i server. Per quanto riguarda
invece i requisiti di protezione l'obiettivo che si ci si è prefissati è quello di inserire almeno 2 VPN Gateway affinchè un host specifico all'interno della rete possa comunicare in maniera
sicura con i server per lo scambio di informazioni riservate. Di conseguenza è stato scelto uno degli host della prima sede, immaginandosi quindi la rete 20.1.0.0/16 come la sede centrale dell'azienda.
La definizione dei requisiti di sicurezza è quindi definibile in input al framework tramite la seguente tabella:

\begin{table}[H]
    \centering
    \small
    \setlength{\tabcolsep}{2pt} % Riduci il padding delle colonne
   \begin{tabular}{ccccccccc}
        \hline
         Policy & IPSrc & IPDst & pSrc & pDst & tProto & Confidentiality & Intregrity & Untrusted nodes\\
        \hline
        Isolation & 20.1.1.1 & 10.0.0.1 & * & * & ANY & // & // & // \\
        Reachability & 20.1.1.1 & 10.0.0.2 & * & * & ANY & // & // & // \\
        \hline
        Isolation & 20.1.2.1 & 10.0.0.1 & * & * & ANY & // & // & // \\
        Reachability & 20.1.2.1 & 10.0.0.2 & * & * & ANY & // & // & // \\
        \hline
        Isolation & 20.1.3.1 & 10.0.0.1 & * & * & ANY & // & // & // \\
        Reachability & 20.1.3.1 & 10.0.0.2 & * & * & ANY & // & // & // \\
        \hline
        Reachability & 20.2.1.1 & 10.0.0.1 & * & * & ANY & // & // & // \\
        Isolation & 20.2.1.1 & 10.0.0.2 & * & * & ANY & // & // & // \\
        \hline
        Reachability & 20.2.2.1 & 10.0.0.1 & * & * & ANY & // & // & // \\
        Isolation & 20.2.2.1 & 10.0.0.2 & * & * & ANY & // & // & // \\
        \hline
        Reachability & 20.2.3.1 & 10.0.0.1 & * & * & ANY & // & // & // \\
        Isolation & 20.2.3.1 & 10.0.0.2 & * & * & ANY & // & // & // \\
        \hline
        Reachability & 20.2.4.1 & 10.0.0.1 & * & * & ANY & // & // & // \\
        Isolation & 20.2.4.1 & 10.0.0.2 & * & * & ANY & // & // & // \\
        \hline
        Reachability & 20.3.1.1 & 10.0.0.1 & * & * & ANY & // & // & // \\
        Reachability & 20.3.1.1 & 10.0.0.2 & * & * & ANY & // & // & // \\
        \hline
        Reachability & 20.4.1.1 & 10.0.0.1 & * & * & ANY & // & // & // \\
        Reachability & 20.4.1.1 & 10.0.0.2 & * & * & ANY & // & // & // \\
        \hline
        Protection & 20.1.1.1 & 10.0.0.2 & * & 22 & ANY & AES-256-CBC & SHA2-256 & 33.33.33.2 \\
        \hline
    \end{tabular}
    \caption{Definizione requisiti di sicurezza della topologia per Demo B}
    \label{tab:tabellaNodiB}
\end{table}

\begin{itemize}
    \item \textbf{Prima coppia di Regole}: Il Web Client C1-1 deve poter essere sempre in grado di raggiungere con almeno un percorso il server S2 e non deve poter raggiungere con alcun percorso il server S1 all'interno della topologia. Per entrambe le regole non ci sono limitazioni di utilizzo sulla porta e sul protocollo di quarto livello, è quindi possibile utilizzare una qualsiasi porta nel range [0-65536] sia per il traffico in entrata che in uscita ed un protocollo a scelta fra UDP e TCP.
    \item \textbf{Seconda coppia di Regole}: Il Web Client C1-2 deve poter essere sempre in grado di raggiungere con almeno un percorso il server S2 e non deve poter raggiungere con alcun percorso il server S1 all'interno della topologia. Per entrambe le regole non ci sono limitazioni di utilizzo sulla porta e sul protocollo di quarto livello, è quindi possibile utilizzare una qualsiasi porta nel range [0-65536] sia per il traffico in entrata che in uscita ed un protocollo a scelta fra UDP e TCP. 
    \item \textbf{Terza coppia di Regole}: Il Web Client C1-1 deve poter essere sempre in grado di raggiungere con almeno un percorso il server S2 e non deve poter raggiungere con alcun percorso il server S1 all'interno della topologia. Per entrambe le regole non ci sono limitazioni di utilizzo sulla porta e sul protocollo di quarto livello, è quindi possibile utilizzare una qualsiasi porta nel range [0-65536] sia per il traffico in entrata che in uscita ed un protocollo a scelta fra UDP e TCP.
    \item \textbf{Quarta coppia di Regole}: Il Web Client C2-1 deve poter essere sempre in grado di raggiungere con almeno un percorso il server S1 e non deve poter raggiungere con alcun percorso il server S2 all'interno della topologia. Per entrambe le regole non ci sono limitazioni di utilizzo sulla porta e sul protocollo di quarto livello, è quindi possibile utilizzare una qualsiasi porta nel range [0-65536] sia per il traffico in entrata che in uscita ed un protocollo a scelta fra UDP e TCP.
    \item \textbf{Quinta coppia di Regole}: Il Web Client C2-2 deve poter essere sempre in grado di raggiungere con almeno un percorso il server S1 e non deve poter raggiungere con alcun percorso il server S2 all'interno della topologia. Per entrambe le regole non ci sono limitazioni di utilizzo sulla porta e sul protocollo di quarto livello, è quindi possibile utilizzare una qualsiasi porta nel range [0-65536] sia per il traffico in entrata che in uscita ed un protocollo a scelta fra UDP e TCP.
    \item \textbf{Sesta coppia di Regole}: Il Web Client C2-3 deve poter essere sempre in grado di raggiungere con almeno un percorso il server S1 e non deve poter raggiungere con alcun percorso il server S2 all'interno della topologia. Per entrambe le regole non ci sono limitazioni di utilizzo sulla porta e sul protocollo di quarto livello, è quindi possibile utilizzare una qualsiasi porta nel range [0-65536] sia per il traffico in entrata che in uscita ed un protocollo a scelta fra UDP e TCP.
    \item \textbf{Settima coppia di Regole}: Il Web Client C2-4 deve poter essere sempre in grado di raggiungere con almeno un percorso il server S1 e non deve poter raggiungere con alcun percorso il server S2 all'interno della topologia. Per entrambe le regole non ci sono limitazioni di utilizzo sulla porta e sul protocollo di quarto livello, è quindi possibile utilizzare una qualsiasi porta nel range [0-65536] sia per il traffico in entrata che in uscita ed un protocollo a scelta fra UDP e TCP.
    \item \textbf{Ottava coppia di Regole}: Il Web Client C3-1 deve poter essere sempre in grado di raggiungere con almeno un percorso il server S1 e con almeno un percorso anche il server S2. Per entrambe le regole non ci sono limitazioni di utilizzo sulla porta e sul protocollo di quarto livello, è quindi possibile utilizzare una qualsiasi porta nel range [0-65536] sia per il traffico in entrata che in uscita ed un protocollo a scelta fra UDP e TCP.
    \item \textbf{Nona coppia di Regole}: Il Web Client C4-1 deve poter essere sempre in grado di raggiungere con almeno un percorso il server S1 e con almeno un percorso anche il server S2. Per entrambe le regole non ci sono limitazioni di utilizzo sulla porta e sul protocollo di quarto livello, è quindi possibile utilizzare una qualsiasi porta nel range [0-65536] sia per il traffico in entrata che in uscita ed un protocollo a scelta fra UDP e TCP.
    \item \textbf{Decima Regola}: Il Web Client C1-1 deve poter comunicare in maniera sicura con il Web Server S2. Il traffico originato da C1-1 può utilizzare qualsiasi porta di uscita per il protocollo di trasporto ma il Server S2 deve ricevere i dati in ingresso unicamente dalla porta 22. È possibile utilizzare sia il protocollo UDP che TCP per il trasporto. Viene specificato infine il nodo Mnt come nodo non sicuro e attraverso il quale il traffico deve passare cifrato.
\end{itemize}

Come ampiamente è stato già discusso per la Demo A, anche in questo caso non è sufficiente definire le regole su Verefoo per assicurarsi di avere una soluzione ottimale, ma è necessario creare un ambiente isolato in grado di testare la correttezza delle configurazioni prodotte da Verefoo. Anche in questo caso è stato scelto
di utilizzare Docker Compose per la realizzazione di una rete virtuale, nella quale ogni host, server, forwarder, firewall e VPN gateway è rappresentato univocamente da un container in esecuzione. Di seguito viene presentata una parte dei servizi definiti all'interno del docker-compose file per istanziare una topologia del genere:

\begin{lstlisting}[style=yaml,caption={Definizione di Services dell'ambiente virtuale DemoB},label=composeDemoA]
    server1:
    container_name: server1
    hostname: server1
    image: endpoint
    cap_add:
      - NET_ADMIN
    command: >-
      sh -c "route del default && route add -net 0.0.0.0 netmask 0.0.0.0 gw
      10.0.0.100 && tail -F anything"
    networks:
      servers:
        ipv4_address: 10.0.0.1
  server2:
    container_name: server2
    hostname: server2
    image: endpoint
    cap_add:
      - NET_ADMIN
    command: >-
      sh -c "route del default && route add -net 0.0.0.0 netmask 0.0.0.0 gw
      10.0.0.199 && tail -F anything"
    networks:
      servers:
        ipv4_address: 10.0.0.2

   forwarder1:
     container_name: forwarder1
     hostname: forwarder1
     image: endpoint
     cap_add:
        - NET_ADMIN
     volumes:
        - './RouterVPNConfig:/mnt:ro'
     command: sh -c "/mnt/staticroutes/forone && tail -F anything"
     networks:
        servers:
            ipv4_address: 10.0.0.100
          forwarder1_vpn3:
            ipv4_address: 130.0.3.2
          forwarder1_fw: 
            ipv4_address: 80.0.1.2
   client1_1:
            container_name: client1_1
            hostname: client1_1
            image: endpoint
            cap_add:
              - NET_ADMIN
            command: >-
              sh -c "route del default && route add -net 0.0.0.0 netmask 0.0.0.0 gw
              20.1.1.100 && tail -F anything"
            networks:
              clients1_1:
                ipv4_address: 20.1.1.1
    
\end{lstlisting}


In questo design, diversamente dal precedente, è possibile notare come i due Web Server condividono la stessa rete, ma viene impostato un default gateway differente. Questa scelta è stata necessaria in quanto 
come è possibile notare nella figura (\ref{fig:AllocationGraphB}) sono presenti due diversi allocation place per collegare i due elementi. Grazie a questa configurazione sarà quindi possibile instanziare Network Security Functions
differenti a seconda del server con cui si comunica. Nelle definizioni d'esempio è presente pure quella di un forwarder, in questo caso seppur l'immagine utilizzata per la costruzione del container è identica a quella di un endpoint generico
durante l'instanziazione del container viene eseguito un comando bash aggiuntivo che carica il file "forone" dalle static route. Tramite ciò è possibile definire tutti i forwarding path per ogni elemento che non influenza le proprietà di sicurezza della rete.\\
Per quanto riguarda i Web Client all'interno della topologia la singola definizione è identica a quella proposta all'interno della Demo A con l'unica maggiore differenza è che è stato scelto l'ip terminante con \textit{".100"} come il default gateway per ogni 
host.\\ \\ 
Entrando più nello specifico, è stato necessario definire delle immagini dalle quali istanziare i container grazie al Docker deamon. Per tutte le immagini si è deciso di utilizzare delle immagini di alpine, una distribuzione Linux estremamente leggera in termini di dimensioni su disco e generalmente sicura, nella quale sono poi installabili eventuali
package aggiuntivi che permettono di creare delle versioni custom dell'immagine a seconda di ciò che vogliamo far svolgere al container. Nel caso degli endpoint la definizione del Dockerfile è la seguente:


\begin{lstlisting}[style=bashstyle, caption={Definizione Dockerfile Endpoint}, label=lst:bash-dockerfileEnd,numbers=left]
    FROM alpine:latest

    RUN apk update
    RUN apk add hping3 --update-cache --repository http://dl-cdn.alpinelinux.org/alpine/edge/testing
    RUN apk add tcpdump
    RUN apk add  net-tools 
    ENV PS1='\h:\w\$ '
    
    CMD ["/bin/sh"];
\end{lstlisting}

Partendo dall'ultima immagine disponibile nel repository generale di Docker viene installato un container alpine, successivamente viene fatto un update dei vari package disponibili e installato il package hping3 che consente di inviare dei pacchetti ICMP/UDP/TCP
e tracciare il loro percorso all'interno della rete utilizzando il comando \textit{"--traceroute"}. Questo package è  puramente necessario per eseguire le operazioni di debug e controllo all'interno della rete. Completano il container i package tcpdump e net-tools
che permettono rispettivamente di osservare le interfacce di rete per osservare i pacchetti in transito e di eseguire comandi per la configurazione di rete come ifconfig, arp. ipmaddr eccetera.\\

Nonostante gli output non siano ancora stati prodotti, dai requisiti di sicurezza ci si aspetta di avere degli allocation places nel quale verranno istanziati Firewall e VPN. Di conseguenza dei Dockerfile sono stati scritti per la futura allocazione di queste Security Network Functions.
Di seguito viene fornito il Dockerfile dei container destinati a diventare Firewall:
\begin{lstlisting}[style=bashstyle, caption={Definizione Dockerfile Firewall}, label=lst:bash-dockerfileEnd,numbers=left]
    FROM alpine:latest

    RUN apk update
    RUN apk add iptables sudo
    RUN apk add tcpdump
    RUN apk add bash
    RUN apk add --no-cache quagga \
     && touch /etc/quagga/zebra.conf \
     && touch /etc/quagga/vtysh.conf \
     && touch /etc/quagga/ripd.conf
    ENV PS1='\h:\w\$ '     
    CMD ["/bin/sh"];
\end{lstlisting}
All'interno di questa configurazione di alpine viene installato il package iptables, che è necessario in quanto permette di definire a livello software un firewall, di monitorare il traffico in ingresso ed uscita dal nodo e di scartare 
eventuali pacchetti che non soddisfano la configurazione. A questo package si aggiunge quagga che è un package in grado di definire delle funzioni di routing avanzate per comunicazioni basate su TCP o IP. All'interno della cartella di installazione di quagga vengono
poi definiti dei file di configurazione necessari ad un corretto funzionamento del software. I packages di tcpdump e bash sono presenti solamente per effettuare operazioni di debug e test.\\

Infine è disponibile una definizione dell'immagine di un ipotetico VPN Gateway:
\begin{lstlisting}[style=bashstyle, caption={Definizione Dockerfile VPN Gateway}, label=lst:bash-dockerfileEnd,numbers=left]
    FROM alpine:latest
    RUN apk update
    RUN apk add  net-tools 
    RUN apk add iptables sudo
    RUN apk add tcpdump
    RUN apk add --update strongswan && \
        rm -rf /var/cache/apk/* && \
        mkdir -p /etc/strongswan
    ENV PS1='\h:\w\$ ' 
    CMD ["/bin/sh"]
\end{lstlisting}

Quest'ultima configurazione oltre ad utilizzare alcuni packages come iptables, tcpdump e net-tools discussi precedentemente installa il software fondamentale per effettuare operazioni di tunneling VPN ovvero strongswan. Questo package è in grado di accettare
le configurazioni che vengono fornite in output dal Translator di Verefoo, velocizzando quindi le operazioni di testing delle configurazioni prodotte.

\section{Output}
\subsection{Topologia di Rete}
Una volta definito tutto il necessario affinchè sia possibile istanziare in sicurezza e in maniera corretta un ambiente virtuale basato sui container Docker, è possibile iniziare a verificare la correttezza della nuova versione del framework analizzando i suoi output.
È importante sottolineare che il numero di allocation places definiti in input è volutamente inferiore a quello necessario per rispettare i vincoli in maniera ottimale. Tramite questa scelta di design è quindi possibile controllare che i nuovi allocation places istanziati dopo 
l'allocazione delle VPN vengano considerati nel calcolo delle soluzioni ottimali.\\
Fornendo quindi l'output definito nel paragrafo precedente la soluzione prodotta in output da Verefoo è la seguente:

\begin{figure}[h]  % 'h' significa che la figura viene posizionata qui
    \centering
    \includegraphics[width=1\textwidth]{Topologia_finale.png} 
    \caption{Verefoo Output}
    \label{fig:VPNDeploy}
\end{figure}

Come ci si poteva aspettare, dati i requisiti di sicurezza definiti sono state allocate tre Network Security Functions. Al centro della topologia è stato inserito un Firewall, che a prima vista sembra essere posizionato correttamente perchè in quella posizione riesce a scartare il traffico proveniente sia dalla
rete 20.1.*.*/16 che quello dalla rete 20.2.*.*/16. Ai lati sinistro e destro sono invece presenti i due VPN Gateway, occorre portare particolare attenzione a quello situato a sinistra in quanto il suo allocation places non era presente in input (figura \ref{fig:AllocationGraphB}), di conseguenza è possibile dedurre che dopo aver
istanziato una Network Security Function la successiva è stata posizionata in uno degli allocation places aggiuntivi formati nella nuova versione del framework.

\subsection{Configurazioni Network Security Functions}
Data la posizione che sembra definire una soluzione plausibilmente corretta, è possibile analizzare anche le configurazioni fornite per ogni funzione di sicurezza.
La seguente è la tabella che descrive la configurazione dei due gateway VPN:

\begin{table}[H]
    \centering
    \begin{tabular}{ccccccc}
        \hline
         Name & Action & IPSrc & IPDst & pSrc & pDst & tProto \\
        \hline
        VPN1 & ACCESS & 20.1.1.1 & 10.0.0.2 & * & 22 & ANY \\
        VPN2 & EXIT & 20.1.1.1 & 10.0.0.2 & * & 22 & ANY \\
        \hline
    \end{tabular}
    \caption{Configurazione dei due VPN gateway Demo B}
    \label{tab:tabella}
\end{table}

Come si può notare, dato l'unico requisito di protezione che era stato definito, solo due gateway sono stati allocati, rispettivamente di accesso per quanto riguarda il gateway presente a sinistra, e di uscita per quello di destra.
In entrambe le configurazioni si può notare che il singolo host dal quale comunicazioni dovranno essere cifrate è la sorgente 20.1.1.1, mentre il nodo di destinazione è il server S2 definito dall'IP 10.0.0.2. Anche i requisiti sul protocollo di trasporto
sono verificati in quanto la porta di destinazione che il server deve ricevere è la 22 che è quella indicata nei requisiti ed inoltre anche nella configurazione non c'è nessun vincolo di protocollo tra TCP e UDP.\\
Per quanto riguarda invece la configurazione del firewall, Verefoo fornisce il seguente output:

\begin{table}[H]
    \centering
    \begin{tabular}{ccccccc}
        \hline
        Default Action & Action & IPSrc & IPDst & pSrc & pDst & tProto \\
        \hline
        ALLOW & DENY & 20.1.*.* & 10.0.0.1 & * & * & ANY \\
        ALLOW & DENY & 20.2.*.* & 10.0.0.2 & * & * & ANY  \\
        \hline
    \end{tabular}
    \caption{Configurazione Firewall Demo B}
    \label{tab:tabella}
\end{table}

Anche in questo caso il framework sembra produrre una soluzione plausibile, in quanto il Firewall è stato configurato in blacklist, permettendo a qualsiasi traffico il transito tranne a quello 
definito nelle regole specifiche del firewall. In questo modo è possibile far comunicare tranquillamente le sottoreti 20.3.1.0/24 e 20.4.1.0/24 in quanto qualsiasi pacchetto da e per i due server S1 ed S2 non verrà
mai scartato dal firewall. Viceversa per quanto riguarda le sottoreti 20.1.0.0/16 e 20.2.0.0/16 il framework non solo ha prodotto delle regole corrette, ma è riuscito anche a raggruppare tutte le definizioni singole sulle reti più piccole in una
definizione più generale grazie alla notazione con \textit{"*"}. Tramite ciò quindi ogni elemento appartenente alla sottorete 20.1.0.0/16 non potrà comunicare con il server S1 ed ogni elemento nella rete 20.2.0.0/16 non potrà comunicare con il server S2.

\subsection{Configurazioni Strongswan}
Nonostante le configurazioni della topologia e delle network security function siano corrette, il framework prodotto allo stato attuale non fornisce una implementazione aggiornata del Translator di Strongswan, non è quindi possibile ottenere automaticamente 
una versione aggiornata dei file di configurazione "swanctl.conf". Per aggirare questo problema, durante lo sviluppo di questa Demo i file di configurazione dei due Gateway sono stati prodotti manualmente a partire dalle configurazioni fornite da Verefoo. \\
Di conseguenza, il seguente snippet di codice rappresenta uno dei due file di configurazione scritti per testare l'ambiente virtuale:

\begin{lstlisting}[language=sh]
    connections {
   site-site {
      local_addrs  = 130.0.4.1
      remote_addrs = 130.0.3.1
      local {
         auth = pubkey
         certs = VpnConfig1Cert.pem
         id = VpnConfig1.strongswan.org
      }
      remote {
         auth = pubkey
         id = VpnConfig2.strongswan.org
      }
      children {
         net-net {
            local_ts = 20.1.1.1/32            
            remote_ts = 10.0.0.2/32            
            start_action = trap|start 
            rekey_time = 5400
            rekey_bytes = 500000000
            rekey_packets = 1000000
            esp_proposals = aes256-sha2_256-modp2048
         }
      }
      version = 2
      mobike = no
      reauth_time = 10800
   }
   
}
\end{lstlisting}

All'interno di questo esempio la connessione site-to-site è definita dalle interfacce dei due VPN gateway con IP 130.0.4.1 e 130.0.3.1. In locale vengono caricati i certificati attraverso
il file VpnConfig1Cert.pem e l'autenticazione viene effettuata tramite chiave pubblica asimmetrica, in maniera simile viene definito l'id e il metodo di autenticazione del gateway remoto.
Infine viene dichiarata una connessione end-to-end tra il client C1-1 definito da 20.1.1.1/32 e il Web Server S2 definito da 10.0.0.2/32. All'interno di questa connessione viene definita anche la proposta
da effettuare per l'incapsulamento in IPSec tramite il protocollo ESP, utilizzando gli algoritmi di aes256-cbc e sha2-256.

\section{Verifiche e Test}

Grazie alle configurazioni prodotte in output dalla nuova versione del framework è possibile verificare che le soluzioni prodotte siano efficaci e corrette.
Per fare ciò, come nella demo precedente è stato predisposto un ambiente virtuale composto da container Docker che simuleranno il comportamento di ogni elemento di rete descritto nella topologia. In questo caso
le verifiche da effettuare all'interno della topologia saranno diverse perchè non sarà necessario dimostrare solo che le comunicazioni fra l'host C1-1 ed il server S2 siano protette tramite crittografia ma è anche
necessario controllare se le comunicazioni fra i client ed i vari host vengono filtrate dal firewall. \\
Partendo dalla verifica del firewall sarà necessario utilizzare due terminali differenti, il primo corrispondente al client C1-1 che come da requisiti dovrebbe essere in grado di comunicare solo con il server S2, isolando tutte le
comunicazioni con S1. Per testare ciò utilizzeremo i seguenti comandi Linux:

\begin{figure}[H] 
    \centering
    \includegraphics[width=1\textwidth]{(1)FirewallDiscard1.png} 
    \caption{Verifica pacchetti scartati dal Firewall per rete 20.1.*.*}
    \label{fig:Verifica1}
\end{figure}

All'interno del terminale C1-1 eseguiamo il comando \textit{"ping -c 5 [ipaddr]"} per verificare le connessioni con i due server presi in esame.
Nel primo caso testiamo sul server dall'IP "10.0.0.2" ovvero S2: come si può notare vengono trasferiti 5 pacchetti col protocollo ICMP e contemporaneamente vengono
ricevuti 5 pacchetti con una statistica di 0\% packet loss. Se invece eseguiamo la stessa operazione con il server dall'IP "10.0.0.1" il risultato sarà opposto, ovvero tutti
i pacchetti inviati vengono scartati non ricevendo alcuna risposta ed il numero di packet loss sarà uguale al 100\%.\\
Considerando il percorso che eseguono i pacchetti, ovvero\\
 $ \text{C1-1} \rightarrow \text{Lb} \rightarrow \text{VPNGateway1} \rightarrow \text{Firewall} \rightarrow \text{Monitor} \rightarrow \text{VPNGateway2} \rightarrow \text{Server}$
è possibile dedurre che l'elemento che scarta i pacchetti e impedisce le comunicazioni è proprio il firewall con la configurazione prodotta da Verefoo.\\
Tuttavia verificare le comunicazioni fra il client C1-1 ed il Server S2 non è sufficiente a verificare il corretto funzionamento del firewall, è infatti necessario assicurarsi che anche
le comunicazioni all'interno della rete 20.2.0.0/16 vengano filtrate quando si prova a comunicare con il server S1. 
\begin{figure}[H] 
    \centering
    \includegraphics[width=1\textwidth]{(2)FirewallDiscard2.png} 
    \caption{Verifica pacchetti scartati dal Firewall per rete 20.2.*.*}
    \label{fig:Verifica2}
\end{figure}

Per verificare anche questa condizione vengono effettuati gli stessi comandi sul terminale del client C2-1, avendo un IP appartenente alla rete 20.2.0.0/16.
Il risultato che viene visualizzato in output è coerente con quello che ci si aspetta, essendo opposto all'output del client C1-1. Le comunicazioni con il server
10.0.0.2 non risultano possibili in quanto ogni pacchetto è stato scartato dal Firewall ed è presente anche qua il 100\% di packet loss, viceversa quando si fa un ping 
sul server 10.0.0.1 corrispondente a S1 ogni pacchetto viene correttamente trasmesso e ricevuto dal server, passando quindi i controlli del firewall. \\
È quindi possibile affermare che le configurazioni riguardanti i requisiti di raggiungibilità e di isolamento prodotte dal framework risultano corrette e con il minimo uso 
di risorse possibili, in quanto grazie ad un unico firewall sono state soddisfatte tutte le caratteristiche fornite in input. \\ \\
La successiva verifica che deve essere effettuata sulla bontà della soluzione prodotta riguarda l'istanziazione e la configurazione dei VPN Gateway nella topologia di output.
In questa istanza specifica solo 2 gateway sono stati allocati, quindi è sufficiente verificare che i pacchetti in transito tra il nodo C1-1 ed il server S2 risultano cifrati
correttamente. Prendendo spunto dai test effettuati nella demo precedente utilizziamo tcpdump nel seguente modo:

\begin{figure}[H] 
    \centering
    \includegraphics[width=1\textwidth]{(3)Firewall_tcpdump_setup.png} 
    \caption{Setup tcpdump per verifiche di sicurezza}
    \label{fig:Verifica3}
\end{figure}

Il nodo preso in analisi è lo stesso in cui viene instanziato il firewall, utilizzando tcpdump è possibile  osservare e monitorare i pacchetti in transito attraverso una delle interfacce
del nodo. L'interfaccia scelta per questo test è "eth4" corrispondente alla connessione fra il firewall ed il primo VPN Gateway. Il risultato aspettato è quindi, come per la prima demo, osservare
il transito dei pacchetti non come semplici ICMP packets ma come ESP packets, ovvero il protocollo utilizzato per l'incapsulamento in IPsec.\\
Per controllare l'output è quindi necessario, una volta impostato il monitoraggio dell'interfaccia, eseguire il comando di ping come fatto per il firewall precedentemente e controllare l'output 
nel terminale del firewall. Di seguito viene proposto un possibile output effettuato: 


\begin{figure}[H] 
    \centering
    \includegraphics[width=1\textwidth]{(4)Encryptedpacket_output.png} 
    \caption{Verifica cifratura dei pacchetti in transito}
    \label{fig:Verifica4}
\end{figure}

Il risultato ottenuto corrisponde alle aspettative descritte precedentemente. È infatti possibile notare come ogni pacchetto venga codificato con il protocollo ESP e venga monitorato con il path specifico del tunnel VPN.
Di conseguenza è possibile affermare che il risultato prodotto da Verefoo è corretto in quanto garantisce la sicurezza in tutto il traffico partente dal nodo C1-1 al server S2. Grazie a questa ennesima verifica anche i requisiti
di sicurezza sono rispettati, confermando la correttezza del framework e della demo prodotta. Di conseguenza si può affermare che anche il terzo ed ultimo obiettivo della tesi è stato portato a termine, producendo una demo funzionante che
mettesse in risalto le caratteristiche della nuova versione del framework.


\chapter{Conclusions} \label{ch:conclusions}

Conclusion and future works
\bibliographystyle{IEEEtran}
\bibliography{bibliography}
%\include{bibliography}

%if appendixes are needed, uncomment the following lines
%\appendix
%\appendixpage
%\include{appendixA}
%\include{appendixB}



\end{document}

