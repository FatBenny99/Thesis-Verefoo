% per commentare una riga mettere % al suo inizio
% per s-commentare una riga (ossia attivarla) togliere il % al suo inizio
%
\documentclass[cucitura%lascia margine per la rilegatura
%,twoside% per stampa fronte-retro (fortemente consigliato per tesi voluminose, opzionale per le altre)
,12pt% font più grande (12pt) rispetto a quello normalmente usato (11pt)
]{toptesi}
%
% Cambiare encoding a piacere; oppure non caricare nessun encoding se si usano
% solo caratteri a 7 bit (ASCII) nei file d'entrata.
%
\usepackage[a-1b]{pdfx}% formato PDF/A, obbligatorio per l'archiviazione delle tesi di Polito
\usepackage[utf8]{inputenc}% IMPORTANTE! usare codifica UTF-8 per le lettere accentate
\usepackage{amsmath, amssymb}
\usepackage{nccmath}
\usepackage{appendix}
\usepackage{longtable}
\usepackage{lscape}
\usepackage{adjustbox}
\usepackage{float}
\usepackage{accsupp}
%
% Commentare le righe seguenti se NON si è specificata l'opzione "pdfa"
\hypersetup{%
    pdfpagemode={UseOutlines},
    bookmarksopen,
    pdfstartview={FitH},
    colorlinks,
    linkcolor={blue},
    citecolor={red},
    urlcolor={blue}
  }
% \documentclass[11pt,twoside,oldstyle,autoretitolo,classica,greek]{toptesi}
% \usepackage[or]{teubner}
%%%%%%%%%%%%%%%%%%%%%%%%%%%%%%%%%%%%%%%%%%%%%%%%%%%%
%


% per inserire uno spazio "fantasma" nella definizione di un'abbreviazione
\usepackage{xspace}

% per inserire un DOI senza problemi coi caratteri "strani" ivi presenti
\usepackage{doi}
\renewcommand{\doitext}{DOI }% originally was "doi:"

% per inserire correttamente le unità di misura SI (incluse quelle binarie)
\usepackage[binary-units]{siunitx}
% se si desidera usare / invece che la potenza -1 per indicare "al secondo"
\sisetup{per-mode=symbol}

% per inserire codice di programmazione complesso
\usepackage{listings}% per inserire codice di programmazione complesso
\lstset{
basicstyle=\ttfamily,
columns=fullflexible,
xleftmargin=3ex,
numbers=none,
breaklines,
breakatwhitespace,
escapechar=`
}

% modify some page parameters
\setlength{\parskip}{\medskipamount}
\advance\voffset -5mm
\advance\textheight 30mm

% riga orizzontale
\newcommand{\HRule}{\rule{\linewidth}{0.2mm}}
% esempio di creazione di semplici abbreviazioni
\newcommand{\ltx}{\LaTeX\xspace}
\newcommand{\txw}{TeXworks\xspace}
\newcommand{\mik}{MikTex\xspace}
\newcommand{\html}{HTML\xspace}
\newcommand{\xhtml}{XHTML\xspace}

% esempio di creazione di un'abbreviazione con un parametro (il cui uso è indicato da #1)
\newcommand{\cmd}[1]{\texttt{#1}\xspace}
% per citare un RFC, es. \rfc{822}
\newcommand{\rfc}[1]{RFC-#1\xspace}
% per citare un file (es. \file{autoexec.bat}) o una URI fittizia (es. \file{http://www.lioy.it/})
% per le URI vere usare \url o \href
\newcommand{\file}[1]{\texttt{#1}\xspace}
% per inserire codice di esempio in-line
\newcommand{\code}[1]{\lstinline|#1|}
% importante per i pathname Windows perché non si può usare \ essendo un carattere riservato di Latex
\newcommand{\bs}{\textbackslash}
% definizione di un termine: formattazione ed inserimento nell'indice
\newcommand{\tdef}[1]{\textit{#1}\index{#1}}
% meta-termine, usato tipicamente nelle definizioni dei tag
\newcommand{\meta}[1]{\textit{#1}}


\definecolor{blond}{rgb}{0.98, 0.94, 0.75}
\definecolor{gray}{rgb}{0.4,0.4,0.4}
\definecolor{darkblue}{rgb}{0.0,0.0,0.6}
\definecolor{cyan}{rgb}{0.0,0.6,0.6}
\definecolor{Maroon}{rgb}{0.5,0.0,0.0}
\definecolor{darkgreen}{rgb}{0.0,0.5,0.0}

%\ExtendCaptions{english}{Abstract}{Acknowledgements}

\lstset{
	numbers=none, 
	numberstyle=\small, 
	numbersep=8pt, 
	frame = single, 
	framexleftmargin=20pt
}

\lstdefinelanguage{XML}
{
	backgroundcolor = \color{blond},
	basicstyle=\ttfamily\footnotesize,
	morestring=[b]",
	moredelim=[s][\bfseries\color{Maroon}]{<}{\ },
	moredelim=[s][\bfseries\color{Maroon}]{</}{>},
	moredelim=[l][\bfseries\color{Maroon}]{/>},
	moredelim=[l][\bfseries\color{Maroon}]{>},
	morecomment=[s]{<?}{?>},
	morecomment=[s]{<!--}{-->},
	commentstyle=\color{DarkOliveGreen},
	stringstyle=\color{blue},
	identifierstyle=\color{red}
}



\begin{document}
%\renewcommand{\lapagina}{\Roman{page}}
\english

\iflanguage{english}{%
	\retrofrontespizio{This work is subject to the Creative Commons Licence}
	\DottoratoIn{PhD Course in\space}
	\CorsoDiLaureaIn{Master of Science in\space}
	\NomeMonografia{Bachelor Degree Final Work}
	\TesiDiLaurea{Master Degree Thesis}
	\NomeDissertazione{PhD Dissertation}
	\InName{in}
	\CandidateName{Candidate}
	\AdvisorName{Supervisors}
	\TutorName{Tutor}
	\NomeTutoreAziendale{Internship Tutor}
	\CycleName{cycle}
	\NomePrimoTomo{First volume}
	\NomeSecondoTomo{Second Volume}
	\NomeTerzoTomo{Third Volume}
	\NomeQuartoTomo{Fourth Volume}
	\logosede[6cm]{PolitoLogo3}% or comma separated list of logos
}{}

\ateneo{}

%%% scegliere la propria facoltà (solo PRIMA dell'AA 2012-2013)
%
%\facolta[III]{Ingegneria dell'Informazione}
%\facolta[IV]{Organizzazione d'Impresa\\e Ingegneria Gestionale}
%\Materia{Remote sensing}% uso sconsigliato

%\monografia{Gestione informatizzata di un magazzino ricambi}% per la laurea triennale
\titolo{Prova}% per la laurea quinquennale e il dottorato
%\sottotitolo{Metodo dei satelliti medicei}% NON obbligatorio, per la laurea quinquennale e il dottorato

\corsodilaurea{Computer Engineering}% per la laurea di primo e secondo livello

\candidato{Benito \textsc{Marra}}% per tutti i percorsi
\relatore{prof.\ Fulvio Valenza}% per la laurea e il dottorato
\secondorelatore{prof.\  Name Surname}% per la laurea magistrale
\terzorelatore{\tabular[t]{@{}l}
	dott.  Name Surname\\[2pt]dott.  Name Surname
	\endtabular}% per la laurea magistrale
%\sedutadilaurea{Agosto 1615}% per la laurea quinquennale
%\sedutadilaurea{\textsc{July} 2019}% per la laurea triennale
\sedutadilaurea{\textsc{Academic~Year} 2023-2024}% per la laurea magistrale
%\annoaccademico{1615-1616}% solo con l'opzione classica
%\annoaccademico{2006-2007}% idem

%\logosede{logopolito}
%
%\chapterbib %solo per vedere che cosa succede; e' preferibile comporre una sola bibliografia
%\AdvisorName{Supervisors}
%\newtheorem{osservazione}{Osservazione}% Standard LaTeX


\hypersetup{
   pdfpagemode={UseOutlines},
   bookmarksopen,
    pdfstartview={FitH},
    colorlinks,
    linkcolor={blue},
    citecolor={green},
    urlcolor={blue}
  }

%
% per numerare e far comparire nell'indice anche le sezioni di quarto livello
%\setcounter{secnumdepth}{4}% section-numbering-depth
%\setcounter{tocdepth}{4}% TOC-numbering-depth (TOC=Table-Of-nt)

%\setbindingcorrection{3mm}

\errorcontextlines=9

\expandafter\ifx\csname StileTrieste\endcsname\relax
    \frontespizio
\else
    \paginavuota
    \tomo
\fi




\sommario


Text of the summary 



\ringraziamenti

Acknowledgement (optional)

%% inserire sempre nella tesi per la laurea di I livello, perché il nome dei tutori non è indicato sul frontespizio.
%Il lavoro descritto in questa monografia è stato svolto sotto la supervisione
%del Prof. Antonio Lioy (tutore accademico)% inserire sempre il nome del tutore accademico
% e dell'Ing. Mario Rossi (tutore aziendale)% inserire solo se la monografia è relativa ad un tirocinio.
%.

%\tablespagetrue % normalmente questa riga non serve ed e' commentata
%\figurespagetrue % normalmente questa riga non serve ed e' commentata

\indici

\listoffigures

\listoftables

\addcontentsline{toc}{chapter}{Listings}
\lstlistoflistings

\clearpage\pagestyle{empty}\mbox{}\clearpage

%\renewcommand{\lapagina}{\arabic{page}}

\mainmatter


\chapter{Introduction} \label{ch:intro}

\section{Thesis objective} 

Benvenuti alla mia tesi figli di puttana!

\section{Thesis description}

description \cite{noms2020}
%other chapters
\chapter{Conclusions} \label{ch:conclusions}

Conclusion and future works

\bibliographystyle{IEEEtran}
\bibliography{bibliography}
%\include{bibliography}

%if appendixes are needed, uncomment the following lines
%\appendix
%\appendixpage
%\include{appendixA}
%\include{appendixB}



\end{document}

