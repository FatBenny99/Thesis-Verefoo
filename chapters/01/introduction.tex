

\hypersetup{
    colorlinks=true,
    linkcolor=blue
}

\chapter{Introduzione} \label{ch:intro}

\section{Obiettivo della Tesi} 

Nell'ultimo decennio diverse tecnologie di rete si sono sviluppate,
creando reti sempre più robuste ed efficienti. In questo scenario,
una tecnologia in particolare, la \textit{"Network Function Virtualization"}(NFV)
ha reso possibile creare delle reti che svolgono le funzioni di sicurezza e
trasporto dei dati tramite la virtualizzazione di quest'ultime. Ciò ha permesso di definire
le \textit{"Software Defined Network"}(SDN) che introducono la possibilità di controllare le operazioni di rete
 tramite software.\\
Sfruttando queste tecnologie è stato sviluppato Verefoo(VErified REfinement and Optimized Orchestration) cioè 
un framework in grado di riuscire ad implementare nei nodi della rete i vari \textit{"Network Security Requirements"}(NSR) in una topologia predefinita e fornita
come input al framework.\\
Gli obiettivi di questa tesi possono essere riassunti nei 3 punti:

\begin{enumerate}
    \item Sistemare e migliorare una demo precedentemente sviluppata per mostrare le potenzialità del framework.
        All'interno di questa demo il framework può accettare solo un determinato requisito di sicurezza di rete ovvero
        la \textit{Protection Property} cioè la possibilità di far passare il traffico crittografato da un nodo ad un altro della topologia
        in maniera sicura. Per fare ciò verranno allocate nella topologia dei VPN Gateway in grado di crittare e decrittare il traffico in ingresso ed in uscita.
    \item Cercare una soluzione per poter integrare le varie versioni di Verefoo. Prima del lavoro che verrà spiegato più avanti il framework era in grado, in un'unica 
        esecuzione, di allocare solo VPN Gateway oppure di allocare solo FireWall per garantire la \textit{Isolation Property} e la \textit{Reachability Property}, mentre
        l'obiettivo che si cerca di raggiungere è di riuscire ad allocare contemporaneamente in un'unica esecuzione del framework sia i VPN Gateway che i Firewall necessari
        al corretto funzionamento della rete.
    \item Produrre una nuova demo, diversa da quella del punto 1 in grado di mostrare le nuove capacità del framework. La demo prodotta dovrà essere in grado di poter allocare,
        in un'unica esecuzione tutti i Firewall e i VPN Gateway definiti dalle proprietà di rete che sono stati forniti come input al framework.
\end{enumerate}

Da questo momento in poi per tutto il documento mi riferiro alla demo da modificare come Demo-A mentre alla demo da implementare come Demo-B.

\section{Descrizione della tesi}

Dopo aver spiegato nel Capitolo \hyperref[ch:intro]{[1]} gli obiettivi ed il lavoro prodotto per raggiungerli, il resto della
tesi è definito nel seguente modo:

\begin{itemize}
    \item Nella prima parte del Capitolo \hyperref[ch:intro]{[2]} si introduce il problema della Network Security Automation e si descrive il framework di Verefoo, ponendo particolare attenzione sul suo funzionamento ad alto e basso livello.
        Nella seconda parte sono descritte  le  definizioni delle Proprietà di sicurezza da passare come input al framework, con una spiegazione dettagliata
        di come queste intervengono nella definizione della topologia finale che verrà fornita come output dal framewrok. \\
        Infine verranno introdotti i grafi che verefoo richiede ed utilizza nella computazione dei vari \textit{NSF}.
    \item Il Capitolo \hyperref[ch:intro]{[3]} definisce l'architettura di docker, specificando la differenza tra usare docker per la virtualizzazione e delle semplici macchine virtuali. Successivamente
          viene fatto un approfondimento sul docker-compose, un tool in grado di poter istanziare più container velocemente tramite script. Nella parte finale viene spiegato come effettuare il networking
          sui container instanziati, come definirlo tramite docker-compose e come testare le comunicazioni in modo efficiente.
    \item Il Capitolo \hyperref[ch:intro]{[4]} descrive i lavori svolti nella Demo-A. Inizialmente viene descritto tramite pezzi di codice lo sviluppo dell'installer prodotto affinchè un qualsiasi utente possa
          utilizzare la demo in maniera pratica ed agile. Nei paragrafi successivi vengono evidenziati i punti critici incontrati, elencando le modifiche apportate affinchè essa possa funzionare correttamente.
          Nell'ultimo paragrafo verranno specificati ulteriori upgrade che si possono inserire nella demo per mettere in mostra in maniera ancora più evidente il lavoro svolto da Verefoo.
    \item Il Capitolo \hyperref[ch:intro]{[5]} contiene un breve approfondimento degli obiettivi 2 e 3 della tesi con una descrizione accurata dei vari passi che sono stati svolti prima della soluzione definitiva.
          In questo capitolo si evidenziano anche le difficoltà che sono emerse lavorando al framework, e verranno proposte alcune soluzioni per poter evitare simili problematiche in futuro.
    \item Il Capitolo \hyperref[ch:intro]{[6]} descrive la soluzione finale scelta ed implementata su Verefoo. In questo capitolo viene quindi spiegato, anche tramite frammenti di codice, gli step
          che il framework eseguirà per produrre in output una rete che soddisfi contemporaneamente tutti i requisiti di sicurezza passati come input. 
    \item Il Capitolo \hyperref[ch:intro]{[7]} descrive lo sviluppo della Demo-B. In un primo momento si mostra la topologia
        di rete scelta da virtualizzare, con la finalità di indicare le nuove funzionalità di verefoo sviluppate al completamento del secondo obiettivo della tesi. Successivamente vengono descritti tutti i passi svolti per implementare la demo, con un commento per il codice che è stato utilizzato. Infine
        si evidenziano anche i limiti della demo prodotta con alcuni  futuri aggiornamenti possibili.
    \item Il Capitolo \hyperref[ch:intro]{[8]} elenca i lavori futuri da svolgere all'interno del framework, la necessità di poter implementare soluzioni
          alternative a quella proposta in questo documento, e i limiti che devono essere superati affinchè il framework possa essere utile in un ambiente reale e non
          solo di testing virtualizzato. Infine vengono descritte le conclusioni del lavoro, con un riassunto generale di tutto ciò che è stato prodotto.
\end{itemize}