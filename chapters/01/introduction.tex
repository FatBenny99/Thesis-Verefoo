

\hypersetup{
    colorlinks=true,
    linkcolor=blue
}

\chapter{Introduzione} \label{ch:intro}

\section{Obiettivo della Tesi} 

Nell'ultimo decennio diverse tecnologie di rete si sono sviluppate,
creando reti sempre più robuste ed efficienti. In questo scenario,
una tecnologia in particolare, la \textit{"Network Function Virtualization"}(NFV)
ha reso possibile creare delle reti che svolgono le funzioni di sicurezza e
trasporto dei dati tramite la virtualizzazione di quest'ultime. Ciò ha permesso di definire
le \textit{"Software Defined Network"}(SDN) che introducono la possibilità di controllare le operazioni di rete
 tramite software.\\
Sfruttando queste tecnologie è stato sviluppato Verefoo(VErified REfinement and Optimized Orchestration) cioè 
un framework in grado di riuscire ad implementare nei nodi della rete i vari \textit{"Network Security Requirements"}(NSR) in una topologia predefinita e fornita
come input al framework.\\

\section{Descrizione della tesi}

Dopo aver spiegato nel Capitolo \hyperref[ch:intro]{[1]} gli obiettivi ed il lavoro prodotto per raggiungerli, il resto della
tesi è definito nel seguente modo:

\begin{itemize}
    \item Nella prima parte del Capitolo \hyperref[ch:verefoo]{[2]} si introduce il problema della Network Security Automation e si descrive il framework di Verefoo, ponendo particolare attenzione sul suo funzionamento ad alto e basso livello.
        Nella seconda parte sono descritte  le  definizioni delle Proprietà di sicurezza da passare come input al framework, con una spiegazione dettagliata
        di come queste intervengono nella definizione della topologia finale che verrà fornita come output dal framewrok. \\
        Infine verranno introdotti i grafi che verefoo richiede ed utilizza nella computazione dei vari \textit{NSF}.
    \item Il Capitolo \hyperref[ch:docker]{[3]} definisce l'architettura di docker, specificando la differenza tra usare docker per la virtualizzazione e delle semplici macchine virtuali. Successivamente
          viene fatto un approfondimento sul docker-compose, un tool in grado di poter istanziare più container velocemente tramite script. Nella parte finale viene spiegato come effettuare il networking
          sui container instanziati, come definirlo tramite docker-compose e come testare le comunicazioni in modo efficiente.
    \item Il Capitolo \hyperref[ch:ThesisObj]{[4]} descrive gli obiettivi posti all'inizio di questo lavoro di tesi. Più specificatamente, per ogni obiettivo presente verranno specificate le modalità e le scelte effettuate per portarlo a termine con una descrizione accurata dei vari passi che sono stati svolti prima della soluzione definitiva.
          Inoltre viene descritto in maniera più profonda rispetto a questo indice la descrizione dei futuri capitoli.
    \item Il Capitolo \hyperref[ch:ThesisObj]{[5]} descrive i lavori svolti nella prima delle due demo di cui questa tesi tratterà. Inizialmente viene descritto tramite pezzi di codice lo sviluppo dell'installer prodotto affinchè un qualsiasi utente possa
          utilizzare la demo in maniera pratica ed agile. Nei paragrafi successivi vengono evidenziati i punti critici incontrati, elencando le modifiche apportate affinchè essa possa funzionare correttamente.
          Nell'ultimo paragrafo infine verranno specificati ulteriori upgrade che si possono inserire nella demo per mettere in mostra in maniera ancora più evidente il lavoro svolto da Verefoo.
    \item Il Capitolo \hyperref[ch:intro]{[6]} descrive i lavori svolti ed implementati su Verefoo. In questo capitolo viene descritto il processo di merge fra le versioni precedentemente esistenti di Verefoo.
          Successivamente verrà quindi spiegato, anche tramite frammenti di codice, gli step
          che il framework eseguirà per produrre in output una rete che soddisfi contemporaneamente tutti i requisiti di sicurezza passati come input.  
          Infine si evidenziano anche le difficoltà che sono emerse lavorando al framework, e verranno proposte alcune soluzioni per poter evitare simili problematiche in futuro.
    \item Il Capitolo \hyperref[ch:intro]{[7]} descrive lo sviluppo della seconda Demo. In un primo momento viene mostrata la topologia
        di rete scelta da virtualizzare, con la finalità di indicare le nuove funzionalità di verefoo sviluppate al completamento del secondo obiettivo della tesi. Successivamente vengono descritti tutti i passi svolti per implementare la demo, con un commento per il codice che è stato utilizzato. Infine
        si evidenziano anche i limiti della demo prodotta con alcuni  futuri aggiornamenti possibili.
    \item Il Capitolo \hyperref[ch:conclusions]{[8]} elenca i lavori futuri da svolgere all'interno del framework, la necessità di poter implementare soluzioni
          alternative a quella proposta in questo documento, e i limiti che devono essere superati affinchè il framework possa essere utile in un ambiente reale e non
          solo di testing virtualizzato. Infine vengono descritte le conclusioni del lavoro, con un riassunto generale di tutto ciò che è stato prodotto.
\end{itemize}