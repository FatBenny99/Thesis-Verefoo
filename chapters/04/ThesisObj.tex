\chapter{Obiettivi della tesi} \label{ch:ThesisObj}

Questo capitolo introduce gli obiettivi di questa tesi, descrivendo studi e metodologie utilizzate al fine di raggiungerli.\\
Nei capitoli precedenti è stato infatti descritto lo stato dell'arte di Docker e Verefoo, che sono i due strumenti principali utilizzati per
svolgere questo lavoro di tesi. Il primo è infatti uno strumento fondamentale per poter garantire un ambiente di testing efficiente e isolato, il secondo
invece è il framework principale nel quale la quasi totalità del lavoro si è svolta. Comprendere l'utilizzo correto dei due elementi è quindi di fondamentale importanza al fine di 
poter capire, continuare e migliorare lo stato attuale di Verefoo. Inoltre, molti dei risultati prodotti precedentemente su Verefoo rispetto a questo lavoro pur essendo corretti non
offrivano alcun modo di mostrare in maniera diretta le innovazioni prodotte ai nuovi utenti che si approcciavano al framework. Parte del lavoro svolto è quindi basato sull'ideare e produrre
dei metodi efficaci e semplici per mostrare all'utente le capacità e caratteristiche di Verefoo.\\
Entrando più nello specifico, gli obiettivi della tesi possono essere definiti dal seguente elenco:


\begin{enumerate}
    \item Come primo obiettivo ci si è focalizzati su una demo già presente all'interno dell'ecosistema. All'interno di questa, tuttavia, diversi elementi all'interno erano considerabili obsoleti
        o scorretti, di conseguenza ci si è posti come scopo principale di questa prima parte correggere e perfezionare la demo per mostrare correttamente le potenzialità del framework.
        Allo stato iniziale, il framework era in grado di accettare solo un determinato requisito di sicurezza di rete ovvero
        la \textit{Protection Property} cioè la possibilità di far passare il traffico crittografato da un nodo ad un altro della topologia
        in maniera sicura. Al fine di garantire ciò vengono allocati nella topologia dei VPN Gateway in grado di poter cifrare il traffico in ingresso e decifrare quello in uscita.
        La topologia proposta utilizzerà uno scenario verosimile a quello che ci si potrebbe aspettare in un'azienda di piccole-medie dimensioni, nella quale al fine di poter garantire
        la correttezza dei requisiti proposti, verranno istanziati 6 VPN Gateway. I lavori svolti per questo obiettivo sono consultabili nel capitolo 5 di questa tesi.\newpage
    \item Una volta terminato il restauro della demo sulle VPN, è emersa la necessità di integrare alle funzionalità già presenti la possibilità di configurare anche i packet filter. Come secondo obiettivo ci si è quindi concentrati per trovare una soluzione al fine di poter integrare le varie versioni di Verefoo. 
        Inizialmente il framework era diviso in differenti branch, due dei quali permettevano rispettivamente l'allocazione solamente dei VPN Gateway o dei Firewall configurati come packet filter per garantire la \textit{Isolation Property} e la \textit{Reachability Property}.
        Il traguardo previsto è quello di creare un ulteriore branch che permettesse la fusione dei due precedentemente descritti. Per ottenere ciò diverse soluzioni sono state esplorate. Inizialmente si è pensato di avere una soluzione mista tramite due versioni del framework attive contemporaneamente che comunicavano fra loro in sequenza,
        per poi passare a soluzioni che permettevano con un solo file jar di svolgere entrambe le funzioni in una sola esecuzione. Anche in questo caso sono stati analizzate entrambe le possibili soluzioni per implementare questo obiettivo, sia istanziando prima i Firewall che i gateway VPN che il viceversa.
        La soluzione finale scelta è stata quella di allocare prima i VPN Gateway e successivamente i Firewall, con delle motivazioni a supporto che verranno estese nel capitolo 6.
    \item Concluso il lavoro sul framework è risultato essenziale trovare un modo per mostrare i risultati ottenuti. L'ultimo obiettivo del lavoro svolto è stato quindi la progettazione, lo sviluppo e l'implementazione di un'altra demo, diversa dalla precedente che mostrasse le nuove potenzialità del framework. \\
        A differenza della prima, che da questo momento verrà definita come Demo-A, la seconda, che chiameremo Demo-B, propone un esempio di topologia di rete molto più complessa e con diverse proprietà di sicurezza aggiuntive. Lo sviluppo di questa ha richiesto, come nella precedente, la realizzazione di un'ambiente virtuale dedicato creato con
        Docker-Compose nel quale mostrare come le varie proprità venissero rispettate. Infine, per agevolare i futuri lavori nel framework è stato prodotto in linguaggio Bash un installer per rendere semplice ed immediato l'installazione del framework. Ulteriori approfondimenti sul codice e le scelte effettuate sono descritte nel capitolo 7.
\end{enumerate}

Come ultima appendice al lavoro svolto ai fini di questa tesi, è infine presente una breve conclusione del lavoro che oltre a fare un riassunto generale sugli obiettivi raggiunti definisce i futuri lavori possibili e suggerisce anche alcuni aggiornamenti e perfezionamenti che possono essere svolti nelle demo e nel framework prodotti.
