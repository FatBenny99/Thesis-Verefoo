\chapter{Conclusioni e Lavori Futuri} \label{ch:conclusions}

Durante lo sviluppo dei lavori di tesi numerosi passi avanti nello sviluppo di Verefoo sono stati effettuati.
Inizialmente, dopo aver studiato approfonditamente il funzionamento e le limitazioni del framework è stata prodotta
una demo consultabile in maniera open source per eventuali utenti che vorrebbero testare e verificare la funzionalità di 
allocazione di tunnel VPN Gateway all'interno di una topologia di rete.
\\

In seconda battuta, diverse modifiche sono state implementate al codice sorgente del software che precedentemente era in grado
di allocare solamente Firewall o solamente tunnel VPN dipendentemente dalla versione. La soluzione prodotta invece è mista e consente
di fornire all'utente topologie di rete che siano in grado di allocare contemporaneamente entrambe le Network Security Functions.
\\ 

Dopo aver testato e corretto eventuali problemi della nuova versione del framework è stata prodotta una demo conclusiva del lavoro svolto,
in grado di mostrare in maniera evidente le nuove funzionalità del merge prodotto. All'interno di quest'ultima viene inoltre mostrato un test
della correttezza del lavoro prodotto, che è stato successivamente documentato e reso disponibile per eventuali pubblicazioni all'interno dei repository
del framework.
\\

Nonostante il lavoro prodotto risulta correttamente testato e completo, è possibile effettuare ulteriori innovazioni per rendere il prodotto disponibile all'utente
ancora più efficiente. Per quanto riguarda le due demo prodotte i nodi configurati come Network Addess Translator all'interno della topologia di rete in input al framework
non vengono istanziati come tali all'interno dell ambiente virtuale, ma sono stati instanziati invece dei nodi placeholder che agiscono come dei semplici forwarder dei pacchetti (con le rispettive static routes).
Di conseguenza utilizzando software come iptables è possibile configurarli come dei NAT veri e propri per simulare in maniera quanto più affidabile possibile la situazione reale descritta in input.
\\
Considerando invece il lavoro svolto sul codice sorgente di Verefoo il framework sembra funzionare nella maggior parte dei casi correttamente, tuttavia una parte ancora non sviluppata all'interno del framework è la 
traduzione e configurazione automatica dei file di StrongSwan da produrre per le soluzioni prodotte. Sarebbe necessario automatizzare anche il meccanismo di configurazione di questi file, così da velocizzare eventuali
test su ambienti virtualizzati e fornire all'utente una primitiva ma corretta configurazione dei tunnel VPN.\\
Inoltre nonostante la versione prodotta durante questo lavoro se ha risposto efficacemente a diversi input ci sono stati alcuni output più complessi di quelli presentati in esempio in questa tesi che hanno portato a delle soluzioni
che erano incomplete o completamente errate, probabilmente dovuto ai vincoli definiti ed imposti su Z3 che conterranno qualche errore di configruazione.\\

Infine l'ultima innovazione che è consigliabile implementare è quella di aggiungere ulteriori funzioni di sicurezza di rete come ad esempio un IDS (Intrusion Detection System) o Web Application Firewall, al fine di garantire più
flessibilità e scelta all'utente nel definire le funzionalità che la rete da configurare deve avere.